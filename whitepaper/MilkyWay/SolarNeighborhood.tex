% ====================================================================
%+
% SECTION:
%    section-name.tex  % eg lenstimedelays.tex
%
% CHAPTER:
%    chapter.tex  % eg cosmology.tex
%
% ELEVATOR PITCH:
%    Explain in a few sentences what the relevant discovery or
%    measurement is going to be discussed, and what will be important
%    about it. This is for the browsing reader to get a quick feel
%    for what this section is about.
%
% COMMENTS:
%
%
% BUGS:
%
%
% AUTHORS:
%    Phil Marshall (@drphilmarshall)  - put your name and GitHub username here!
%    Will Clarkson (@willclarkson)
%
%-
% ====================================================================

\section{The Solar Neighborhood}
\def\secname{solarneighborhood}\label{sec:\secname} % For example, replace "keyword" with "lenstimedelays"

\noindent{\it Author Name(s)} % (Writing team)

% This individual section will need to describe the particular
% discoveries and measurements that are being targeted in this section's
% science case. It will be helpful to think of a ``science case" as a
% ``science project" that the authors {\it actually plan to do}. Then,
% the sections can follow the tried and tested format of an observing
% proposal: a brief description of the investigation, with references,
% followed by a technical feasibility piece. This latter part will need
% to be quantified using the MAF framework, via a set of metrics that
% need to be computed for any given observing strategy to quantify its
% impact on the described science case. Ideally, these metrics would be
% combined in a well-motivated figure of merit. The section can conclude
% with a discussion of any risks that have been identified, and how
% these could be mitigated.

A short preamble goes here. What's the context for this science
project? Where does it fit in the big picture?
The science projec

% --------------------------------------------------------------------

\subsection{Target measurements and discoveries}
\label{sec:keyword:targets}

Generally the cases in this science area require deep photometry,
particularly (though not exclusively) at red bandpasses, and
sufficient per-image precision and distribution of observing times in
at least one filter to measure parallax and/or proper motion. The
main-survey area is not sufficient for this science area: dedicated
observations are required for all science cases to fill in angular
coverage (apart from photometric variability of cool dwarfs).

Three representative science cases identified (more are in the Phase I
Science Roadmap):
\begin{itemize}

\item Volume-complete astrometric sample of extended Solar Neighborhood: requires parallax and proper motion, deep multi-filter photometry to classify objects.; 
\item Discovery of ultracold brown dwarfs (late-T and Y); requires very deep photometry in (i,z,Y) 

\item Photometric variability of cool dwarfs; requires sufficient time baseline sampling to chart variability; photometry at other bands to classify object. 

\end{itemize}

%Describe the discoveries and measurements you want to make.

%Now, describe their response to the observing strategy. Qualitatively,
%how will the science project be affected by the observing schedule and
%conditions? In broad terms, how would we expect the observing strategy
%to be optimized for this science?


% --------------------------------------------------------------------

\subsection{Metrics}
\label{sec:keyword:metrics}

Organized by broad science case within this area:
\begin{itemize}
\item Volume-complete astrometric sample of extended Solar Neighborhood:
  \begin{itemize}
  \item Bias and scatter in distance distribution of extended Solar Neighborhood objects;
  \item Bias and scatter in individual source $T_{eff}, \log(g)$~as a
    function of spectral type;
  \item Bias and scatter in population constrants (e.g. mass function slope) for candidate mass functions of local populations;
    \item Fraction of local-volume population recovered {\it and} correctly identified (e.g. spectral subtype error?).
  \end{itemize} 

\item Discovery of ultracold brown dwarfs (late-T/Y):
  \begin{itemize}
  \item Bias and scatter in individual source $T_{eff}, \log(g)$~as a
    function of spectral type;
    \item Bias and scatter in the brown dwarf/exoplanet (?) mass function parameters.
    \item Fraction of objects within $X$~pc from the Sun recovered {\it and} correctly identified (e.g. spectral subtype error?).
  \end{itemize}


\item Photometric variability of cool dwarfs
\begin{itemize}
  \item {\it (Note: hierarchical metric?) E.g. classification of X\% of starspot variability in local volume, which calls Variability Metric $Y$ as a method.}
\end{itemize}

\end{itemize}

%Quantifying the response via MAF metrics: definition of the metrics,
%and any derived overall figure of merit.

% --------------------------------------------------------------------

\subsection{OpSim Analysis}
\label{sec:keyword:analysis}

OpSim analysis: how good would the default observing strategy be, at
the time of writing for this science project?


% --------------------------------------------------------------------

\subsection{Discussion}
\label{sec:keyword:discussion}

One useful output of MAF for Solar Neighborhood work, will be the
extent to which the small part of the sky in front of crowded regions
in this general science area will make an impact. 

Comparison of the science metrics for cases when Galactic Plane
observations are / are not included as part of the OpSim run, will
directly address the degree to which LSST should consider Solar
Neighborhood science in its galactic plane strategy.

%Discussion: what risks have been identified? What suggestions could be
%made to improve this science project's figure of merit, and mitigate
%the identified risks?


% ====================================================================

\navigationbar
