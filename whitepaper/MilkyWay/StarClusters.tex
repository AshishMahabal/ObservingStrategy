% ====================================================================
%+
% SECTION:
%    section-name.tex  % eg lenstimedelays.tex
%
% CHAPTER:
%    chapter.tex  % eg cosmology.tex
%
% ELEVATOR PITCH:
%    Explain in a few sentences what the relevant discovery or
%    measurement is going to be discussed, and what will be important
%    about it. This is for the browsing reader to get a quick feel
%    for what this section is about.
%
% COMMENTS:
%
%
% BUGS:
%
%
% AUTHORS:
%    Phil Marshall (@drphilmarshall)  - put your name and GitHub username here!
%    Will Clarkson (@willclarkson)
%
%-
% ====================================================================

\section{Star Clusters}
\def\secname{starclusters}\label{sec:\secname} % For example, replace "keyword" with "lenstimedelays"

\noindent{\it Author Name(s)} % (Writing team)

% This individual section will need to describe the particular
% discoveries and measurements that are being targeted in this section's
% science case. It will be helpful to think of a ``science case" as a
% ``science project" that the authors {\it actually plan to do}. Then,
% the sections can follow the tried and tested format of an observing
% proposal: a brief description of the investigation, with references,
% followed by a technical feasibility piece. This latter part will need
% to be quantified using the MAF framework, via a set of metrics that
% need to be computed for any given observing strategy to quantify its
% impact on the described science case. Ideally, these metrics would be
% combined in a well-motivated figure of merit. The section can conclude
% with a discussion of any risks that have been identified, and how
% these could be mitigated.

A short preamble goes here. What's the context for this science
project? Where does it fit in the big picture?

[Star cluster leads]

%Two broad science areas {\bf [more?]}: (i) Better understanding of
%stellar populations within clusters; (ii) star clusters as tracers of
%star formation in the Galaxy, including donation to the Plane by
%dissolution

% --------------------------------------------------------------------

\subsection{Target measurements and discoveries}
\label{sec:keyword:targets}


\begin{itemize}

\item Formation history and evolution of the Milky Way as traced by star clusters

\item Better understanding of star cluster stellar populations; stellar mass function, metallicity, ages

\end{itemize}

%Describe the discoveries and measurements you want to make.

%Now, describe their response to the observing strategy. Qualitatively,
%how will the science project be affected by the observing schedule and
%conditions? In broad terms, how would we expect the observing strategy
%to be optimized for this science?


% --------------------------------------------------------------------

\subsection{Metrics}
\label{sec:keyword:metrics}

Organized by broad science case within this area:
\begin{itemize}
\item Milky Way formation and evolution as traced by star clusters (focus below on tidal dissolution $\rightarrow$~contribution of stars to the Disk by clusters):
  \begin{itemize}
    \item Radius $R$ from cluster center at which LSST becomes unable to kinematically identify cluster members (figure might be: [x,y,z] = [cluster $M_v$, concentration parameter, $R$]; rerun for different locations in the Galaxy?
      \item Sensitivity (quantified as surface brightness limit?) to tidal streams between clusters? Perhaps go broader, e.g. mass ratio or $M_v$~for tidally interacting clusters for which LSST becomes insensitive to existence of tidal tail?
        \item LSST's sensitivity to azimuthal asymmetries in the light (or mass) profile of clusters ``near'' the plane
  \end{itemize} 

\item Stellar populations within clusters; mass function, metallicity, ages
  \begin{itemize}
    \item Radius of avoidance for LSST per-filter for a catalog of
      known clusters (in-plane and out)
      \item Distance limit at which LSST becomes insensitive to stars above mass $m_{lower}$ 
  \end{itemize}

\end{itemize}

(Metrics note: all of the above are likely to be hierarchical, calling
some parameterization of photometric sensitivity vs crowding as a
method.)

Need some sort of metric for the science return on the ensemble of
globular clusters observed by LSST. If LSST doesn't go deeper than X
at latitude $|b| < Y$, what is the impact on the science?

%Quantifying the response via MAF metrics: definition of the metrics,
%and any derived overall figure of merit.

% --------------------------------------------------------------------

\subsection{OpSim Analysis}
\label{sec:keyword:analysis}

OpSim analysis: how good would the default observing strategy be, at
the time of writing for this science project?


% --------------------------------------------------------------------

\subsection{Discussion}
\label{sec:keyword:discussion}

One useful output of MAF for Solar Neighborhood work, will be the
extent to which the small part of the sky in front of crowded regions
in this general science area will make an impact. 

Comparison of the science metrics for cases when Galactic Plane
observations are / are not included as part of the OpSim run, will
directly address the degree to which LSST should consider Solar
Neighborhood science in its galactic plane strategy.

%Discussion: what risks have been identified? What suggestions could be
%made to improve this science project's figure of merit, and mitigate
%the identified risks?


% ====================================================================

\navigationbar
