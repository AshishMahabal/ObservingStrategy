% ====================================================================
%+
%
% SECTION NAME:
%    \secname.tex
%
% CHAPTER:
%    ???.tex
%
%
% COMMENTS:
%
%
% BUGS:
%
%
% AUTHORS:
%   Ohad Shemmer, Timo Anguita, Niel Brandt, Gordon Richards,
%   Scott Anderson(?), Phil Marshall(?)
%-
% ====================================================================
\clearpage
\section{AGN Science}
\def\secname{agn}\label{sec:\secname}

\credit{ohadshemmer}
\credit{tanguita}
\credit{nielbrandt}
\credit{GordonRichards}
\credit{ScottAnderson}

% This section discusses the potential effects of the LSST observing
% strategy on AGN science. In short, there appears to be a consensus
% among the AGN and galaxies communities that AGN science will benefit
% from the most uniform cadence in terms of even sampling for each band
% and uniform sky coverage. It is also expected that any reasonable
% perturbation to the nominal LSST observing strategy will have mostly
% minor effects on AGN science. This section attempts to identify all
% the areas of AGN science that may be affected by the observing strategy
% and to point out the metrics that can be used to quantify any potential
% effect. Since the total number of metrics that must be quantified is
% quite large, and the effects are likely small in most cases, the goal
% of this section is to identify potential ``killers'' that may undermine
% key AGN research areas. For example, certain perturbations may reduce
% significantly the number of ``interesting'' AGNs, such as $z>6$ quasars,
% lensed quasars, or transient AGNs. Another example is photometric
% reverberation mapping which is one of LSST's greatest advantages for
% AGN research but is also very sensitive to the cadence; care must be
% taken to ensure that the observing strategy does not undermine the
% ability to make the best use of this method.

\subsection{AGN Selection and Census}
\label{sec:\secname:selection}

\noindent About $10^7 - 10^8$ AGNs will be selected in the main LSST
survey using a combination of criteria, split broadly into four
categories: colors, astrometry, variability, and multiwavelength
matching with other surveys. The LSST observing strategy will affect
mostly the first three of these categories.

{\bf Colors:} The LSST observing strategy will determine the depth in
each band, as a function of position on the sky, and will thus affect
the color selection of AGNs. This will eventually determine the AGN
$L-z$ distribution and, in particular, may affect the identification
of quasars at $z\gtsim 6$ if, for example, $Y$-band exposures will
not be sufficiently deep.

{\bf Variability:} AGNs can be effectively distinguished from (variable)
stars, and from quiescent galaxies, by exhibiting certain characteristic
variability patterns (e.g., \citet{ButlerandBloom2011}). Non-uniform
sampling may ``contaminate'' the variability signal of AGN candidates.

{\bf Astrometry:} AGNs will be selected among sources having zero
proper motion, within the uncertainties. The LSST cadence may affect
the level of this uncertainty in each band, and may therefore affect
the ability to identify (mostly fainter) AGN.
%
Differential chromatic refraction (DCR), making use of the astrometric
offset a source with emission lines has with respect to a source with
a featureless power-law spectrum, can help in the selection of AGNs
and in confirming their photometric redshifts
\citep{KaczmarczikEtal2009}. The DCR effect is more pronounced at
higher airmasses. Therefore, it'd be advantageous to have at least one
visit, per source, at airmass greater than about 1.4. AGN selection
and photometric redshift confirmation may be affected since the LSST
cadence will affect the airmass distribution, in each band, for each
AGN candidate.

\subsection{AGN Clustering}
\label{sec:\secname:clustering}

\noindent Measurements of the spatial clustering of AGNs with respect
to those of quiescent galaxies can provide clues as to how galaxies
form inside their dark-matter halos and what causes the growth of
their supermassive black holes (SMBHs). The impressive inventory of
LSST AGNs will enable the clustering, and thus the host galaxy halo
mass, to be determined over the widest range ranges of cosmic epoch
and accretion power.
%
The LSST cadence will not only affect the overall AGN census and its
$L-z$ distribution, but also the depth in each band as a function of
sky position that can directly affect the clustering signal.

\subsection{AGNs and the Time Domain}
\label{sec:\secname:time}

{\bf AGN Variability:} A variety of AGN variability studies will be
enabled by LSST. These are intended to probe the physical properties
of the unresolved inner regions of the central engine. Relations will
be sought between variability amplitude and timescale vs. $L$, $z$,
$\lambda_{\rm eff}$, color, multiwavelength and spectroscopic
properties, if available. The LSST sampling is expected to provide
high-quality power spectral density (PSD) functions for a large number
of AGNs; these can be used to constrain the SMBH mass and accretion
rate/mode. Furthermore, LSST AGNs exhibiting excess variability over
that expected from their luminosities will be further scrutinized as
candidates for lensed systems having unresolved images with the excess
(extrinsic) variability being attributed mainly to microlensing.

Photometric reverberation mapping (PRM), measuring the time-delayed
response of either the flux of the broad emission line region (BELR)
lines to the flux of the AGN continuum or between the continuum flux
in one (longer wavelength) band to the continuum flux in another
(band with shorter wavelength), will be one of the cornerstones of
AGN research in the LSST era (e.g., \citet{Chelouche2013};
\citet{CheloucheandZucker2013}; \citet{CheloucheEtal2014};
\citet{EdelsonEtal2015}; \citet{FausnaughEtal2015}). For example,
LSST is expected to deliver BELR line-continuum time delays in
$\sim10^5-10^6$ sources, which is unprecedented when compared to
$\sim50-100$ such measurements conducted via the traditional, yet
much more expensive (per source) spectroscopic method. Sources in
the deep-drilling fields (DDFs) will benefit from the highest quality
PRM time-delay measurements given the factor of $\sim10$ denser
sampling. The PRM measurements will probe the size and structure of
the accretion disk and BELR, in a statistical sense, and may provide
improved SMBH mass estimates for sources that have at least
single-epoch spectra.

The PRM method is very sensitive to the sampling in each band,
therefore the ability to derive reliable time delays can be affected
significantly by the LSST cadence. The best results will be obtained
by having the most uniform sampling equally for each band.
Additionally, there is a trade-off between the number of DDFs and
the number of time delays that PRM can obtain \citep{CheloucheEtal2014}.
For example, an increase in the number of DDFs, with similarly dense
sampling in each field, can yield a proportionately larger number of
high-quality time delays, down to lower luminosities, but at the
expense of far fewer time delays (for relatively high luminosity
sources) in the main survey.

Potentially periodic AGN variability, leading to tentative discoveries
of binary SMBHs (e.g., \citet{GrahamEtal2015}) may also benefit from
uniform sampling. Over the ten-year survey, LSST will be sensitive
to periods of a few days up to $\sim3$~yr.

{\bf Time Delays in Gravitationally Lensed Quasars:} This aspect is
discussed in detail in the lens time delays section
(\autoref{sec:lenstimedelays}).

{\bf AGN Size and Structure with Microlensing:} Microlensing due to
stars projected on top of individual lensed quasar images produce
additional magnification. Using the fact that the Einstein radii of
stars in lensing galaxies closely match the scales of different
emission regions in high-redshift AGNs (micro-arcseconds), analyzing
microlensing induced flux variations statistically on individual
systems allows us to measure ``sizes'' of AGN regions.
%
Assuming a thermal profile for accretion disks, sizes in different
emission wavelengths will be probed and as such, constraints on the
slope of this thermal profile will be placed. Given the sheer number
of lensed systems that LSST is expected to discover ($\sim8000$),
this will allow us to stack systems for better constraints and
hopefully determine the evolution of the size and profile. Due to the
typical relative velocities of lenses, microlenses, observers (Earth)
and source AGN, the microlensing variation timescales are between
months to a few decades.

The quasar microlensing optical depth is $\sim1$, so every lensed
quasar should be affected by microlensing at any given point in time.
However, measurable variability can occur on longer timescales.
\citet{MosqueraandKochanek2011} studied all known lensed quasars.
They found the median timescale between high magnification events
(Einstein crossing time scales) in the observed $I$-band is of the
order of $\sim20$~yr (with a distribution between 10 and 40~yr).
However, the source crossing time (duration of a high magnification
event) is $\sim7.3$~months (with a distribution tail up to 3~yr).
This basically means that out of all the lensed quasar {\em images}
(microlensing between images is completely uncorrelated) about half
of them will be quiescent during the 10~yr baseline of LSST. However,
since the typical number of lensed images is either two or four, it
means that, statistically, in every system, one (for doubles) or two
(for quads) high magnification events should be observed in 10~yr of
LSST monitoring.

Note that, the important cadence parameter is the source crossing time,
as it is the length of the event to be as uniformly sampled as
possible. The 7.3 months crossing time is the median for the observed
$i$-band, but this time would be significantly shorter for bluer bands:
for a thermal profile with slope
$\alpha: R_\lambda \propto \lambda^\alpha$ implies source crossing time
$t_{\rm s} \propto \lambda^{1/\alpha} \rightarrow
t_u=t_i \times (\lambda_{\rm u} / \lambda_{\rm i})^{1/\alpha}$. For a
Shakura-Sunyaev slope of $\alpha=0.75$ this would correspond to
$7.3 \times (3600/8140)^{4/3}$ months which is $\approx 2.5$ months in
the $u$-band.

In terms of the cadence, at least three evenly sampled data points per
band within two to three months would be preferred to be able to map
the constraining high magnification event(?). Hopefully uniformly
spaced. Very tight cadence (e.g., DDFs) would increase the constraints
significantly. However, since lensed quasars are not that common, this
smaller area would mean only a few ($\sim80$?) suitable systems
monitored in the DDFs.
%
Regarding the season length, the ``months'' timescale of high
magnification events very likely means that we can/will miss high
magnification events in the season gaps, at least in the bluer bands.
%
Killer: observations spread on timescales larger than 3 months(?).
This would likely miss the high magnification events. In those cases
we could perhaps consider close consecutive photometric bands as
equivalent accretion disk regions, however this would mean weaker
constraints on the thermal profile.
%
Important Note: all this science needs to be done on lensed quasars
with measured or very short time delays to remove the intrinsic
variability signal, which might significantly reduce the sample.

{\bf Microlensing Aided Reverberation Mapping:} Given that
microlensing mostly affects continuum emission rather than BELR line
emission, microlensing may enable disentangling the BELR line plus the
continuum emission in single photometric bands, allowing the use of
single broad band PRM measurements \citep{SluseandTewes2014}. As with
the two-band PRM method discussed above, the denser (and the longer)
the sampling, the more accurate are the constraints that can be
obtained for the time delays.

{\bf Transient AGN and TDEs:} This aspect is discussed in detail in
the non-periodic variables section (\autoref{sec:variables}).

% --------------------------------------------------------------------

\subsection{Metrics}
\label{sec:\secname:metrics}

% Quantifying the main impacts on AGN science via MAF metrics, including the effects
% of additional cadence factors such as the number of DDFs
% and MC fields, or different dithering patterns, definition of the metrics,
% and any derived overall figure of merit.

{\bf AGN Selection:} Need to compute the mean $Y$ band magnitude
across the sky for the nominal OpSim. Compare this magnitude to the
one required for identifying $\geq1000$ quasars at $z\geq6$.
Currently, for enigma\_1189, the single-epoch 5-sigma depth in $Y$
is 22.36 mag, and for the final co-added 5-sigma depth is $Y=24.4$.
These limits are deeper than the original predictions (see the AAS
poster from 2013:
http://www.lsst.org/sites/default/files/221-RC-247.10-AAS_shemmer.pptx.pdf).

{\bf PRM:} Need to compute the average and the dispersion in the
number of visits, per band, across the sky for the nominal OpSim
(during the entire survey). Since PRM works best for uniform sampling,
need to compare the distributions of the number of visits in each
band, averaged across the sky, and identify ways to minimize any
potential differences between these distributions. By running PRM
simulations, identify the 1) minimum number of visits (in any band)
that can yield any meaningful BELR-continuum lag estimates, and 2) the
largest difference in the number of visits between two different bands
that can yield any meaningful BELR-continuum lag estimates. Repeat
these simulations by doubling the nominal number of DDFs. Assess the
number of meaningful BELR-continuum time delays that can be obtained
with the nominal OpSim, and point out potential perturbations in the
cadence to improve the number and quality of such time delays.

{\bf Microlensing:} Need to compute the dispersion in the time gap
between visits in the same band, across the sky, in order to assess
the fraction of microlensing events that might be missed (on top of
seasonal gaps).

% Ideas for Metrics:
% detection - how many can LSST detect based on the luminosity function
% (depends on the depth in each band for single epoch and coadd)
% (how will this change with each DR)? @ohadshemmer

% classification - How many of these will we actually classify as quasars?
% non-simultaneous colors. variability of QSOs (how does depend on
% cadence/baseline/seasonal gaps?)

% microlensing - convolve microlensing timescales for QSOs we already know
% about. how many of the high magnification events do we get? How bright?
% @tanguita

% QPOs - Will the cadence and duration give you the proper range on the
% power spectrum to detect QPOs with a given mass, spin, and L/Ledd?
% (Bob Wagoner)

% How is this interdependent with the photo-z metrics?

% --------------------------------------------------------------------

\subsection{Discussion}
\label{sec:\secname:discussion}

% Discussion: what risks have been identified? What suggestions could be
% made to improve the figures of merit, and mitigate the identified risks?
% What ``tweaks'', if any, can be proposed to the nominal LSST observing strategy
% in order to help achieve key AGN science goals?

Some additional considerations/thoughts that came up during the workshop:

1) Assuming we will have 10 DDFs, perhaps one of those fields could be
sampled more heavily than the others and would be visited nightly (or
even more frequently, e.g., from $\sim1$ to $\sim1000$ min) per band.
This can be justified by the fact that
a) very few AGNs have been monitored at these frequencies on such a
long baseline, leaving room for discovery, and b) this may probe
intermediate-mass black holes ($\sim10^4 - 10^5$~$M_{\odot}$) via PRM or
PSDs. Good candidate fields are the CDF-S and the MCs.
Need to suggest a strategy for an OpSim (or commissioning).

2) Need to assess the effects of the LSST cadence on the ability to
detect periodic AGNs and quasi-periodic oscillations (QPOs) in AGNs.

3) Needs from OpSim: in order to have informative metrics, we need to have
accurate model light curves that can reproduce how the fiducial light
curves appear in different bands, at different inherent luminosities,
and at different redshifts.

% DDF - we need longer duration OpSims in these fields. (Bob Wagoner)
% (https://github.com/LSSTScienceCollaborations/ObservingStrategy/blob/master/opsim/README.md)

% commissioning opportunity - one field, one night, one filter (u or g).
% 15/30 sec exposures. (Bob Wagoner)
% (https://github.com/LSSTScienceCollaborations/ObservingStrategy/blob/master/commissioning/README.md)

\todo{}{Compare the current science content with the AGN chapter in the
Science Book as well as with the Ivezic et al. overview paper
(http://arxiv.org/pdf/0805.2366v4.pdf).}

\todo{}{Compare the $Y$-band (and other bands) depths, single
epoch and final co-added, from  enigma\_1189 with other OpSims.}

\todo{}{Assess the effect of non-simultaneous colors on AGN selection.}

\todo{}{Based on the current OpSim, need to specify the magnitude limits
at the highest airmass and assess the limitations of the DCR method in
the $L-z$ plane. Should check this with MAF and if indeed AM <= 1.4, need
to add a request in:
https://github.com/LSSTScienceCollaborations/ObservingStrategy/tree/master/opsim
}

\todo{}{Assess whether, e.g., a pair of $\sim2.5$ min exposures (i.e., $\sim10$
times longer than the standard exposures) at airmass $\sim2$ would get deep
enough for useful DCR constraints for a large fraction of the AGNs. This may
be a non-negligible perturbation of the expected 56-184 visits per band, and
may even be impossible given current upper limits on exposure times, but
this would help improve photo-z's for galaxies and SNe too.}

\todo{}{For PRM and microlensing: obtain distributions (mean and dispersion)
of the number of visits, per band, across the sky for the nominal OpSim
(during the entire survey).}

\todo{}{Add a discussion about blazars and LSST cadence (e.g., Isler+15?).
Would blazar science benefit from a specific cadence requirement?}

\todo{}{Check the Astrometry chapter and assess the astrometric precision
for AGN selection. For example, we may require very good depth at least in
the 1st and 10th year of the survey.}

\todo{}{Indicate the properties (frequency range and sampling) for obtaining
optimal PSDs required for QPO detection (given MBH, spin, and L/LEdd). Based on
current OpSim, assess the potential of discovering QPOs in the main survey
and in the DDFs.}

\navigationbar
