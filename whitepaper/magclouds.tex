% ====================================================================
%+
% NAME:
%    section-name.tex
%
% ELEVATOR PITCH:
%    Explain in a few sentences what the relevant discovery or
%    measurement is going to be discussed, and what will be important
%    about it. This is for the browsing reader to get a quick feel
%    for what this section is about.
%
% COMMENTS:
%
%
% BUGS:
%
%
% AUTHORS:
%    David Nidever (@dnidever)
%    Knut Olsen (@knutago)
%-
% ====================================================================

\section{ The Magellanic Clouds }
\def\secname{MCs}\label{sec:\secname}

\noindent{\it David L. Nidever, Knut Olsen} % (Writing team)

% This individual section will need to describe the particular
% discoveries and measurements that are being targeted in this section's
% science case. It will be helpful to think of a ``science case" as a
% ``science project" that the authors {\it actually plan to do}. Then,
% the sections can follow the tried and tested format of an observing
% proposal: a brief description of the investigation, with references,
% followed by a technical feasibility piece. This latter part will need
% to be quantified using the MAF framework, via a set of metrics that
% need to be computed for any given observing strategy to quantify its
% impact on the described science case. Ideally, these metrics would be
% combined in a well-motivated figure of merit. The section can conclude
% with a discussion of any risks that have been identified, and how
% these could be mitigated.

%A short preamble goes here. What's the context for this science
%project? Where does it fit in the big picture?
The Magellanic Clouds have always had outsized importance for astrophysics.  They are critical steps in the cosmological distance ladder, they are a binary galaxy system with a unique interaction history, and they are laboratories for studying all manner of astrophysical phenomena.  They are often used as jumping-off points for investigations of much larger scope and scale; examples are the searches for extragalactic supernova prompted by the explosion of SN1987A and the dark matter searches through the technique of gravitational microlensing.  More than 17,000 papers in the NASA ADS include the words ``Magellanic Clouds'' in their abstracts or as part of their keywords, highlighting their importance for a wide variety of astronomical studies.

An LSST survey that did not include coverage of the Magellanic Clouds and their periphery would be tragically incomplete.  LSST has a unique role to play in surveys of the Clouds.  First, its large $A\Omega$ will allow us to probe the thousands of square degrees that comprise the extended periphery of the Magellanic Clouds with unprecedented completeness and depth, allowing us to detect and map their extended disks, stellar halos, and debris from interactions that we already have strong evidence must exist (REFS).  Second, the ability of LSST to map the entire main bodies in only a few pointings will allow us to identify and classify their extensive variable source populations with unprecedented time and areal coverage, discovering, for example, extragalactic planets, rare variables and transients, and light echoes from explosive events that occurred thousands of years ago (REFS).  Finally, the large number of observing opportunities that the LSST 10-year survey will provide will enable us to produce a static imaging mosaic of the main bodies of the Clouds with extraordinary image quality, an invaluable legacy product of LSST.

We propose two distinct mini-surveys to meet the goals of LSST Magellanic Clouds science:
\begin{itemize}
\item A mini-survey covering the 2700$\deg^2$ with $\delta < -60$ to the standard LSST single-exposure depth and to stacked depths of XXX, with cadence sufficient to detect and measure light curves of RR Lyrae stars 
\item A mini-survey covering $\sim$250$\deg^2$ of the main bodies of the Clouds with cadence sufficient to detect exoplanet transits and other variable objects; a subset of these images should be taken with seeing of $0.5\arcsec$, with stacked depth reaching the confusion limits in the Clouds
\end{itemize}

These surveys will support several important scientific goals:


%David's text
Two main overarching science themes:
\begin{enumerate}
\item {\bf Galaxy formation evolution}: The study of the formation and evolution of the Large and Small
  Magellanic Clouds (LMC and SMC, respectively), especially their interaction with each other and the Milky Way.
  The Magellanic Clouds (MCs) are a unique local laboratory for studying the formation and evolution of
  dwarf galaxies in exquisite detail.  LSST's large FOV will be able to map out the three-dimensional
  structure, metallicity and kinematics in great detail.
\item {\bf Stellar astrophysics \& Exoplanets}:  The MCs have been used for decades to study stellar
  astrophysics, microlensing and other processes.  The fact that the objects are effectively all at a single
  known distance makes it much easier to study them than in, for example, the Milky Way.  LSST will extend
  these studies to fainter magnitudes, higher cadence, and larger area.
\end{enumerate}

Many different types of objects and measurements with their own cadence ``requirements'' will fall into
these two broad categories (with some overlap).  These will be outlined in the next section.

A very important aspect of the ``galaxy evolution'' science theme is not just the cadence but also the
sky coverage of the Magellanic Clouds ``mini-survey''.  A common misunderstanding is that the MCs only
cover a few degrees on the sky.  That is, however, just the central regions of the MCs akin to the thinking
of the Milky Way as the just the bulge.  The full galaxies are actually much larger with LMC stars
detected at $\sim$21$^{\circ}$ ($\sim$18 kpc) and SMC stars at $\sim$10$^{\circ}$ ($\sim$11 kpc) from their
respective centers.  The extended stellar debris from their interaction likely extends to even larger
distances.  Therefore, to get a complete picture of the complex strucure of the MCs will require a
mini-survey that covers $\sim$2000 deg$^2$.  At this point, it not entirely clear how to include this
into the metrics.  Note, that for the second science case this is not as much of an issue since the large
majority of the relevant objects will be located in the high-density, central regions of the MCs.


% --------------------------------------------------------------------

\subsection{Target measurements and discoveries}
\label{sec:keyword:targets}

%Describe the discoveries and measurements you want to make.
%
%Now, describe their response to the observing strategy. Qualitatively,
%how will the science project be affected by the observing schedule and
%conditions? In broad terms, how would we expect the observing strategy
%to be optimized for this science?

\begin{enumerate}

\item Deep Color Magnitude Diagrams
%  -Deep CMDs, just a matter of number of visits
%  -do the full SMASH (and relevant DES area) with full spatial coverage, at least to SMASH depths, smaller
%  number of epochs, ~5 sigma at gri~25
% Knut thoughts: I think we want to make sure that we get 1 mag below old turnoff out to 100 kpc in ugriz with 10sigma precision, i.e. ugriz~25


\item Proper Motions
%-Proper Motions, cadence not as much of an issue, just more epochs
%  bulk proper motion
%  LMC spiral motion, streaming motions
%  internal velocity dispersion

%\item Parallaxes
%-Parallaxes, also mostly a function of nubmer of epochs
%  bulk distances
%  internal distance spread

\item Variable stars

%-Variables, RR Lyrae, Cepheids might be too bright, dwarf cepheids/scuti good, many more of them.
%   especially good for getting the 3D structure (out to large distances) of the MCs
%   -eclipsing binaries (get very accurate distances, see OGLE paper), pulsating WDs, CVs, T Tauri stars

\item Transients
%-Transients, dwarf novae

\item Transiting Exoplanets
% -Transiting planets

\item Astrometric binaries
%-Astrometric binaries

\item Gyrochronology
%-Gychronology, need to get periods of the dwarfs, gives age information

\item Astroseismology
%-Astroseismology, dwarfs/giants, giants vary by a couple percent and on "longer" timescales, but
%    probably too bright for LSST, OGLE probably has best data for those. however LSST might be able to do
%    asteroseismology of giants to larger distances, measure masses/ages of halo giants!
%    dwarfs are harder because they vary less and need more higher frequency observations

\end{enumerate}

% --------------------------------------------------------------------

\subsection{Metrics}
\label{sec:keyword:metrics}

Quantifying the response via MAF metrics: definition of the metrics,
and any derived overall figure of merit.


% --------------------------------------------------------------------

\subsection{OpSim Analysis}
\label{sec:keyword:analysis}

OpSim analysis: how good would the default observing strategy be, at
the time of writing for this science project?


% --------------------------------------------------------------------

\subsection{Discussion}
\label{sec:keyword:discussion}

Discussion: what risks have been identified? What suggestions could be
made to improve this science project's figure of merit, and mitigate
the identified risks?


% ====================================================================

\navigationbar
