
% ====================================================================
%+
% NAME:
%    section-name.tex
%
% ELEVATOR PITCH:
%    Explain in a few sentences what the relevant discovery or
%    measurement is going to be discussed, and what will be important
%    about it. This is for the browsing reader to get a quick feel
%    for what this section is about.
%
% COMMENTS:
%
%
% BUGS:
%
%
% AUTHORS:
%    Phil Marshall (@drphilmarshall)  - put your name and GitHub username here!
%-
% ====================================================================

\section{Probing Planet Populations with LSST}
\def\secname{periodicvariables}\label{sec:\secname}

\noindent{\it Author Name(s)} % (Writing team)

% This individual section will need to describe the particular
% discoveries and measurements that are being targeted in this section's
% science case. It will be helpful to think of a ``science case" as a
% ``science project" that the authors {\it actually plan to do}. Then,
% the sections can follow the tried and tested format of an observing
% proposal: a brief description of the investigation, with references,
% followed by a technical feasibility piece. This latter part will need
% to be quantified using the MAF framework, via a set of metrics that
% need to be computed for any given observing strategy to quantify its
% impact on the described science case. Ideally, these metrics would be
% combined in a well-motivated figure of merit. The section can conclude
% with a discussion of any risks that have been identified, and how
% these could be mitigated.

This section describes the unique discovery space for extrasolar planets with LSST.

\subsection{Planets In Relatively Unexplored Environments}


% --------------------------------------------------------------------

\subsection{OpSim Analysis}
\label{sec:keyword:analysis}



% --------------------------------------------------------------------

\subsection{Discussion}
\label{sec:keyword:discussion}

% ====================================================================

\navigationbar
