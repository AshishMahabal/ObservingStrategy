
% ====================================================================
%+
% NAME:
%    section-name.tex
%
% ELEVATOR PITCH:
%    Explain in a few sentences what the relevant discovery or
%    measurement is going to be discussed, and what will be important
%    about it. This is for the browsing reader to get a quick feel
%    for what this section is about.
%
% COMMENTS:
%
%
% BUGS:
%
%
% AUTHORS:
%    Phil Marshall (@drphilmarshall)  - put your name and GitHub username here!
%-
% ====================================================================

\section{Probing Planet Populations with LSST}
\def\secname{periodicvariables}\label{sec:\secname}

\noindent{\it Author Name(s)} % (Writing team)

% This individual section will need to describe the particular
% discoveries and measurements that are being targeted in this section's
% science case. It will be helpful to think of a ``science case" as a
% ``science project" that the authors {\it actually plan to do}. Then,
% the sections can follow the tried and tested format of an observing
% proposal: a brief description of the investigation, with references,
% followed by a technical feasibility piece. This latter part will need
% to be quantified using the MAF framework, via a set of metrics that
% need to be computed for any given observing strategy to quantify its
% impact on the described science case. Ideally, these metrics would be
% combined in a well-motivated figure of merit. The section can conclude
% with a discussion of any risks that have been identified, and how
% these could be mitigated.

This section describes the unique discovery space for extrasolar planets with LSST.

\subsection{Planets In Relatively Unexplored Environments}
A large number of exoplanets have been discovered over the past twenty years, with over 1500 exoplanets now confirmed. These discoveries are primarily the result of three detection methods; radial velocity discoveries measure a period and minimum mass of the planet, transiting discoveries measure a radius ratio between the planet and the star and can be combined with RV measurements to constrain the mass and determine a density; microlensing discoveries measure planet mass but represent non-periodic events.

The Kepler mission has an additional almost 4000 planet candidates. While these planet candidates have not been confirmed, the sample is significant enough that planet characteristics can be studied statistically, including radius and period distributions and planet occurrence rates. LSST will add to previous transiting planet searches by observing stellar populations that have generally not been well-studied by previous transiting planet searches, including cluster stars, the galactic bulge, red dwarfs, and the magellanic clouds. Most known exoplanets have been found relatively nearby in the Galaxy, as exoplanet systems with measured distances have a median distance of around 80 pc, and 80\% of these systems are within 320 pc (exoplanets.org). LSST is able to recover transiting exoplanets at much larger distances, including in the galactic bulge and the Large Magellanic Cloud, allowing for measurements of planet occurrence rates in these other stellar environments (Lund http://arxiv.org/pdf/1408.2305v2.pdf, Jacklin http://arxiv.org/pdf/1503.00059v2.pdf). Red dwarfs have often been underrepresented in searches that have focused on solar-mass stars, however red dwarfs are plentiful, and better than 1 in 7 are expected to host earth-sized planets in the habitable zone (Dressing http://arxiv.org/pdf/1501.01623v2.pdf).

While most of the sky that LSST will survey will be at much lower cadences than transiting planet searches employ, a sufficient understanding of the LSST efficiency for detecting planets combined with the large number of targets may still provide significant results. Additionally, the multiband nature of LSST provides an extra benefit, as exoplanet transits are achromatic while many potential astrophysical false positives, such as binary stars, are not.

% --------------------------------------------------------------------

\subsection{OpSim Analysis}
\label{sec:keyword:analysis}



% --------------------------------------------------------------------

\subsection{Discussion}
\label{sec:keyword:discussion}

% ====================================================================

\navigationbar
