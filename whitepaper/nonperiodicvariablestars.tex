
% ====================================================================
%+
% NAME:
%    section-name.tex
%
% ELEVATOR PITCH:
%    Explain in a few sentences what the relevant discovery or
%    measurement is going to be discussed, and what will be important
%    about it. This is for the browsing reader to get a quick feel
%    for what this section is about.
%
% COMMENTS:
%
%
% BUGS:
%
%
% AUTHORS:
%    Phil Marshall (@drphilmarshall)  - put your name and GitHub username here!
%-
% ====================================================================

\section{Non-periodic Variable Stars}
\def\secname{variables}\label{sec:\secname}

\noindent{\it Author Name(s)} % (Writing team)

% This individual section will need to describe the particular
% discoveries and measurements that are being targeted in this section's
% science case. It will be helpful to think of a ``science case" as a
% ``science project" that the authors {\it actually plan to do}. Then,
% the sections can follow the tried and tested format of an observing
% proposal: a brief description of the investigation, with references,
% followed by a technical feasibility piece. This latter part will need
% to be quantified using the MAF framework, via a set of metrics that
% need to be computed for any given observing strategy to quantify its
% impact on the described science case. Ideally, these metrics would be
% combined in a well-motivated figure of merit. The section can conclude
% with a discussion of any risks that have been identified, and how
% these could be mitigated.


Some variable star types are not strictly periodic.  These include multi-period stars, for which Fourier analysis
may be useful, but only if the underlying frequencies are at least
critically sampled.  Irregularly variable stars may have so little
repeatable structure that  neither phase stacking nor harmonic
analysis is very useful, but patterns may become evident over
observing intervals of years or decades.

Non-periodic variable stars may benefit from complete phase coverage
over a single cycle, or other time interval  of of interest, repeated
in consecutive intervals, or in intervals distributed over the survey.
Variable stars for which thorough sampling of limited duration is
required, including eruptive variables, are considered below with
transients.

Active galactic nuclei are mentioned here as well, as they may vary as transients and/or variable, in most cases with no or only weak periodicity.

% --------------------------------------------------------------------

\subsection{Targets and Measurements}
\label{sec:\secname:targets}

The class of non-periodic variables includes a heterogeneous
assortment of objects and phenomena.

\begin{center}
\begin{tabular}{| p{5cm} | p{10cm} |}
\hline Variable Type & Examples of target science\\
\hline
Long Period Variables & Pulsation modes, internal structure, evolution\\
Multimode pulsation & Pulsation mechanisms, internal structure\\
Semi-regular variables & Pulsation mechanisms, convection \\
Pulsating irregular variables & Chaotic dynamics \\
Epsilon Aurigae systems & Circumstellar material, dark companions\\
FU Ori systems & Accretion events, jets\\
Young Stellar Objects & Accretion, jets, disks, binarity, flaring, rotation, spots, magnetic phenomena\\
Active galactic nuclei & Galaxy evolution, reverberation mapping, black hole physics\\
 \hline \end{tabular}
 \end{center}

In each case, the observational challenge is to discover and then to
characterize the targets, utilizing the power of the LSST survey to
increase by orders of magnitude the number of well-studied targets
known.  Most of the targets in the table have variation time scales of
$\simeq$ 1 week or greater, and will receive sampling commensurate
with the time scale of variation under a natural LSST cadence
($\sim$800 visits over 10 years).  Where a higher sampling rate is
needed, these will need customized attention to the time scale and the
number and duration of sampled intervals.

% --------------------------------------------------------------------

\subsection{Metrics}
\label{sec:\secname:metrics}

\begin{center}
\begin{tabular}{| p{5cm} |p{10cm} |}
\hline Metric & Description\\
\hline
Non-periodic variables & Histogram of median visit series length vs maximum visit spacing within the series\\
  \hline \end{tabular}
 \end{center}


% --------------------------------------------------------------------

\subsection{OpSim Analysis}
\label{sec:\secname:analysis}

Current simulations provide reasonable sampling ($\sim$2 samples per time constant) for variables that change brightness on a time scale of $>$1 week.  For faster variations, an enhanced sampling rate should be studied.


% --------------------------------------------------------------------

\subsection{Discussion}
\label{sec:\secname:discussion}

Special cadences offer the opportunity to extend LSST studies to non-periodic phenomena with time scales $\leq$1 week, rather than the $>$1 week that is naturally achieved with a uniform survey.


The need for contemporaneous color information has not been addressed, and needs consideration, as with novel targets and non-repeating signals, it may not be possible to infer color relations.
% ====================================================================

\navigationbar
