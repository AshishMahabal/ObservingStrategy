

% ====================================================================
% commands for stand-alone printing
%\documentclass[11pt,headsepline,cleardoubleempty,twoside,openright]{scrbook}
%\usepackage{SciBook}
%\begin{document}
% ====================================================================

% ====================================================================
%+
% NAME:
%    section-name.tex
%
% ELEVATOR PITCH:
%    Explain in a few sentences what the relevant discovery or
%    measurement is going to be discussed, and what will be important
%    about it. This is for the browsing reader to get a quick feel
%    for what this section is about.
%
% COMMENTS:
%
%
% BUGS:
%
%
% AUTHORS:
%    Phil Marshall (@drphilmarshall)  - put your name and GitHub username here!
%-
% ====================================================================

\section{ Rolling Cadence }
\label{sec:keyword} % For example, replace "keyword" with "lenstimedelays"

\noindent{\it Author Name(s)} % (Writing team)

% This individual section will need to describe the particular
% discoveries and measurements that are being targeted in this section's
% science case. It will be helpful to think of a ``science case" as a
% ``science project" that the authors {\it actually plan to do}. Then,
% the sections can follow the tried and tested format of an observing
% proposal: a brief description of the investigation, with references,
% followed by a technical feasibility piece. This latter part will need
% to be quantified using the MAF framework, via a set of metrics that
% need to be computed for any given observing strategy to quantify its
% impact on the described science case. Ideally, these metrics would be
% combined in a well-motivated figure of merit. The section can conclude
% with a discussion of any risks that have been identified, and how
% these could be mitigated.

With a total of ~800 visits spaced approximately uniformly over 10 years, and distributed among 6 filters, 
it is not clear that LSST can offer the sufficiently dense sampling in time for study of transients with typical durations less than or ~1 month.
This is particularly a concern for key science requiring well-sampled SNIa light curves.  Rolling cadences stand out as a
general solution that can potentially enhance sampling rates by 2X or more, on part of the sky all of the time and all of the sky some of the time,
while maintaining a sufficient uniformity for survey objectives that require it.

% --------------------------------------------------------------------
% --------------------------------------------------------------------
\subsection{ Supernovae and Rolling Cadence}
\label{sec:keyword} % For example, replace "keyword" with "lenstimedelays"

\noindent{\it Author Name(s)} % (Writing team)

Supernovae as a science topic are addressed in another chapter.
In this section, the demands of SN are used to directly constrain or
orient the rolling cadence development.

% --------------------------------------------------------------------

\subsection{ Fast Transients and Rolling Cadence}
\label{sec:keyword} % For example, replace "keyword" with "lenstimedelays"

\noindent{\it Author Name(s)} % (Writing team)

Fast transients as a science topic are addressed in another chapter.
In this section, the demands of fast transients are used to directly constrain or
orient the rolling cadence development.

% --------------------------------------------------------------------
\subsection{ Trades and Compromises for Rolling Cadences}
\label{sec:keyword} % For example, replace "keyword" with "lenstimedelays"


What desirable survey parameters will be traded against rolling cadence benefits, how to tune the cadence, and 
perspective.

% ====================================================================

% ====================================================================
% commands for stand-alone printing
% \bibliographystyle{apj}
% \bibliography{references}
% \end{document}
% ====================================================================
