\chapter[The Galaxy]{The Galaxy}
\def\chpname{galaxy}\label{chp:\chpname}

\noindent {\it
Will Clarkson, Kathy Vivas, Peregrine McGehee, and others to follow
}

% --------------------------------------------------------------------

\section{Introduction}
\def\secname{intro}\label{sec:\secname}

% WIC 2015-08-21 15:00: lots of edits to the intro text to refocus the
% document a bit more usefully.

LSST will significantly advance Milky Way science, on lengthscales
from the galactic halo and local volume, right down to sensitive
surveys for faint nearby objects to uncover the true state of the
Solar Neighborhood.
%The broad reach of the Milky Way science
%cases is nicely summarized in Section 2.1.4 of Ivezic et al. (2008
%arXiv 0805.2366).
%\begin{itemize}
%\item What is the detailed structure and accretion history of the Milky Way?
%\item What are the fundamental properties of all the stars within 300 pc of the Sun?
%\end{itemize}
Much more detail about most of these science cases, and specific
science questions to be answered, can be found in the LSST Science
Book (particularly chapters 6 and 7) and Ivezic et al. (2008 arXiv
0805.2366, in particular Sections 2.1.4 and 4.4); in this chapter we
(aim to) present metrics that will allow various observing strategies
to be quantitatively assessed in terms of their impact on science
cases falling under the general rubric of ``The Milky Way.''  

Our intention is to decouple the technical issue of coverage on-sky
from the science accomplished; for example, if a science case includes
regions both inside and outside of the ``Wide-Fast-Deep'' (WFD)
survey, then the metric for a particular science case should produce a
quantitative assessment of the science outcome no matter what the
distribution of pointings on the sky actually was. As a result, issues
to do with areal coverage (for example, whether a census of all nearby
brown dwarfs that neglects ``low-latitude'' fields towards the inner
Galaxy, is scientifically unacceptable) should in theory fall out as a
result of the metrics assessment process.

This chapter is organized as follows. In Section \ref{sec:GeneralMW}
we point out some general considerations specific to much of the Milky
Way science area that we recommend MW scientists bear in mind when
considering metrics. Because of the large diversity of science cases,
Section \ref{sec:SummaryTableMW} presents the essential aspects of
each science case, expressed in terms of observational goals and
tracer population. Science cases dominated by in-plane observations
are presented first in Table X, as the observing strategy for in-plane
observations is currently far from resolved. Then Table Y in this
section presents the vital statistics for science cases covered by
Wide-Fast-Deep. Sections
\ref{sec:MW_spatial_structure}-\ref{sec:MW_localvolume} then present
the metrics and their evaluations for MW science cases. Finally,
section \ref{sec:MW_future_work} points the way to anticipated future
improvements to this living document in light of expected precursor data.

\section{Some general considerations for Milky Way metrics}
\def\secname{GeneralMW}\label{sec:\secname}

Many of the Milky Way science cases involve Galactic Plane
observations almost exclusively. At the time of writing (late August
2015), even the broad parameters of in-plane observations remain to be
decided (for example the number of exposures per-filter).

Before launching into the specifics of metrics per-science case, we
point out some general factors that are likely to obtain for Milky Way
science (all over the sky), which we encourage the reader to
incorporate into their thinking when considering possible
metrics. [Language will change as metrics are produced.]

We currently envisage these to be implemented as methods that a given
science metric would call as part of its chain of evaluation.

{\it 1. Observing the ``foreground:''} By the standards of LSST's
Wide-Fast-Deep survey, many if not most of the objects of interest to
Milky Way science are close enough that they will saturate under the
Wide-Fast-Deep cadence, or will be impacted by bright foreground
objects. With LSST's saturation limit at 15 second exposure in the
neighborhood of $r \sim 17$~({\bf need confirmation!}), metrics should
include in their chain of evaluation, some sensitivity to at least the
following implications of foreground observation:

\begin{itemize}
  \item The upper limit on brightness for which measurements can be made that are sufficiently accurate for a given science case;
  \item The loss in discovery efficiency due to charge bleeds from objects unrelated to the targets for a given science case.
\end{itemize}

The discovery efficiency metric may correlate with existing
first-order metrics already presented elsewhere; for example, the
range of position angles for a field will likely correlate with the
discovery efficiency in the presence of charge bleeds, as a wider
range of position-angles will reduce the number of exposures in which
a given faint target lies underneath the charge bleed from one very
bright foreground object. Or, a given dithering strategy might
increase discovery efficiency due to pathological, very close, very
bright objects being moved into chip-gaps during some of the dithers.

(One suggested observing configuration for the Plane to mitigate the
impact of both of these factors, is to split the exposures per pair
into unequal-length exposures; perhaps $(1 \times 1s + 1\times 10s)$,
to ensure that nearly every object has an unsaturated exposure in
nearly every field. Although we do not wish to suggest a large array
of observing strategies at this stage, we do recommend that this
option be simulated for in-plane observations.)

{\it 2: Crowding and seeing:} Metrics for in-plane science must
incorporate the impact of spatial confusion on both photometric and
astrometric measurements. Both of these depend on seeing. More work
remains to be done on the level of sophistication necessary in these
considerations; for example, when considering relative proper motion
precision, the spatial density of reference stars at similar
brightness to the object whose proper motion is desired (to mitigate
magnitude-dependent PSF effects like the ``fatter-brighter'' effect)
will in principle impact the proper motion precision attainable. The
size of the impact of this effect on proper motion precision should be
determined.

{\it 3: Relative and absolute metrics:} Because even the first-order
observation parameters for in-plane observations are somewhat
unconstrained, metrics should be sensitive to differences in overall
allocation as well as by comparison to the ideal strategy within an
allocation. For example, the way in which observations are distributed
within a time baseline is not very impactful to many science cases if
that baseline is only three years long for OpSim algorithmic reasons!
(For example, OpSim might move all the exposures in a galactic plane
science case into the first three years to finish short projects
early, which would be a disaster for cases that require a long time
baseline.)

\section{Summary table of Milky Way science cases}
\def\secname{SummaryTableMW}\label{sec:\secname}

Here we present a quick-look summary of the Milky Way science cases
to-date, with measurement type and tracer population indicated where
appropriate. This communicates the importance of certain objects and
certain regimes to Milky Way science. Given the considerations
outlined in Section \ref{sec:GeneralMW}, this overview is split in two
tables: Table 1 shows science cases for the five main science cases we
have identified for metric development, while while Table Y summarizes
the rest in terms of tracers and simple scaling relations (the
majority of which can safely assume SRD performance numbers).

{\bf All those bullet points from v1 of this chapter are condensed and summarized in these tables.}

[Table 1 - measurements and tracers for the five main science cases requiring new metrics.]

[Table 2 - Milky Way science cases not in Table 1; scaling relations from other metrics.]

({\it Hierarchical metrics (e.g. github Issue \#79):} since others are
developing detailed metrics for fine-grained issues of detectability,
in this Chapter we focus our candidate metrics mostly on high-level
metrics that might call the variability metrics as functions.) 

The following sections describe observing metrics for Milky Way
science cases. These have signficant (but not exclusive)
observation-sets in or towards the Plane. If your science case is not
mentioned below, it should be mentioned concisely in Table 2.


% Subsection filenames all start with ``MW'' in violation of Phil's
% advice to avoid chapter-specific names. This is to avoid namespace
% collision between ``microlensing'' for AGN and ``microlensing'' in
% the MW, for example!

Section \ref{sec:MW_spatial_structure}: Spatial Structure of the Milky Way Galaxy, including the plane, bulge, halo and local volume;

Section \ref{sec:MW_metallicity_mapping}: Galactic photometric metallicity mapping; 

Section \ref{sec:MW_SFH}: Star formation history of the Milky Way;

Section \ref{sec:MW_Dust}: Dust in the Milky Way plane;

Section \ref{sec:MW_microlensing} Microlensing;

% There may be one more section here: stellar populations of interest
% with unknown scale-heights. Chuck Claver's white dwarf pulsator
% project might be of interest here, among others.

%{\it Milky Way science cases but mostly in Wide-Fast-Deep:}

%Section \ref{sec:solarneighborhood} The Solar Neighborhood;

%Section \ref{sec:starclusters} Star Clusters in the Milky Way;

%Section \ref{sec:halostructure} Halo structure and populations;

%Section \ref{sec:localvolume} The Local Volume

[As of 2015-08-21, the .tex files for these subsections are mostly
  empty shells. Content to follow.]

[To be resolved: how to ensure Solar Neighborhood science is best represented. Does this deserve a sixth section?]

\input{MilkyWay/MW_spatial_structure.tex}
% ====================================================================
%+
% SECTION:
%    section-name.tex  % eg lenstimedelays.tex
%
% CHAPTER:
%    chapter.tex  % eg cosmology.tex
%
% ELEVATOR PITCH:
%    Explain in a few sentences what the relevant discovery or
%    measurement is going to be discussed, and what will be important
%    about it. This is for the browsing reader to get a quick feel
%    for what this section is about.
%
% COMMENTS:
%
%
% BUGS:
%
%
% AUTHORS:
%    Phil Marshall (@drphilmarshall)  - put your name and GitHub username here!
%-
% ====================================================================

\section{Metallicity and age mapping in the Milky Way}
\def\secname{MW_metallicity_mapping}\label{sec:\secname} % For example, replace "keyword" with "lenstimedelays"

\noindent{\it Author Name(s)} % (Writing team)

% This individual section will need to describe the particular
% discoveries and measurements that are being targeted in this section's
% science case. It will be helpful to think of a ``science case" as a
% ``science project" that the authors {\it actually plan to do}. Then,
% the sections can follow the tried and tested format of an observing
% proposal: a brief description of the investigation, with references,
% followed by a technical feasibility piece. This latter part will need
% to be quantified using the MAF framework, via a set of metrics that
% need to be computed for any given observing strategy to quantify its
% impact on the described science case. Ideally, these metrics would be
% combined in a well-motivated figure of merit. The section can conclude
% with a discussion of any risks that have been identified, and how
% these could be mitigated.

A short preamble goes here. What's the context for this science
project? Where does it fit in the big picture?

% --------------------------------------------------------------------

\subsection{Target measurements and discoveries}
\label{sec:keyword:targets}

Describe the discoveries and measurements you want to make.

Now, describe their response to the observing strategy. Qualitatively,
how will the science project be affected by the observing schedule and
conditions? In broad terms, how would we expect the observing strategy
to be optimized for this science?


% --------------------------------------------------------------------

\subsection{Metrics}
\label{sec:keyword:metrics}

Quantifying the response via MAF metrics: definition of the metrics,
and any derived overall figure of merit.


% --------------------------------------------------------------------

\subsection{OpSim Analysis}
\label{sec:keyword:analysis}

OpSim analysis: how good would the default observing strategy be, at
the time of writing for this science project?


% --------------------------------------------------------------------

\subsection{Discussion}
\label{sec:keyword:discussion}

Discussion: what risks have been identified? What suggestions could be
made to improve this science project's figure of merit, and mitigate
the identified risks?


% ====================================================================

\navigationbar

% ====================================================================
%+
% SECTION:
%    section-name.tex  % eg lenstimedelays.tex
%
% CHAPTER:
%    chapter.tex  % eg cosmology.tex
%
% ELEVATOR PITCH:
%    Explain in a few sentences what the relevant discovery or
%    measurement is going to be discussed, and what will be important
%    about it. This is for the browsing reader to get a quick feel
%    for what this section is about.
%
% COMMENTS:
%
%
% BUGS:
%
%
% AUTHORS:
%    Phil Marshall (@drphilmarshall)  - put your name and GitHub username here!
%-
% ====================================================================

\section{Star Formation History of the Milky Way}
\def\secname{MW_SFH}\label{sec:\secname} % For example, replace "keyword" with "lenstimedelays"

\noindent{\it Author Name(s)} % (Writing team)

% This individual section will need to describe the particular
% discoveries and measurements that are being targeted in this section's
% science case. It will be helpful to think of a ``science case" as a
% ``science project" that the authors {\it actually plan to do}. Then,
% the sections can follow the tried and tested format of an observing
% proposal: a brief description of the investigation, with references,
% followed by a technical feasibility piece. This latter part will need
% to be quantified using the MAF framework, via a set of metrics that
% need to be computed for any given observing strategy to quantify its
% impact on the described science case. Ideally, these metrics would be
% combined in a well-motivated figure of merit. The section can conclude
% with a discussion of any risks that have been identified, and how
% these could be mitigated.

Summary: this important topic does not seem to have been developed in
previous versions of the LSST science book or Ivezic et
al. (2008). LSST gives the opportunity to survey extensive areas
around star formation regions in the Southern hemisphere. Among
others, it would allow to study the Initial Mass Function down to the
sub-stellar limit across different environments. Young stars are
efficiently identified by their variability.

A short preamble goes here. What's the context for this science
project? Where does it fit in the big picture?

% --------------------------------------------------------------------

\subsection{Target measurements and discoveries}
\label{sec:keyword:targets}

In order to assess the ability of LSST to 1) identify and 2) classify
YSO we need to quantify the variability timescales and amplitudes of
both Class I/II (stars with disks) and Class III (WTTS). Inclusion of
eruptive variables (FUor/Exor) is appropriate as well - see section
8.10.2 in the Science Book.

In brief, WTTS are quasi-periodic with amplitudes of 0.1 to 0.3 mag
and periods 1 to ~15 days - so comparable to gamma Dor stars (see
Figure 8.17 in the SB). Given the temporal evolution of cool spots, a
period recovery analysis such as shown for RRL stars (se FIgure 8.20
in the SB) is likely difficult. The embedded systems and CTTS are
irregular variables but shown have distinctive colors due to
extinction + UB/blue excess arising from to accretion shocks.


%Describe the discoveries and measurements you want to make.

%Now, describe their response to the observing strategy. Qualitatively,
%how will the science project be affected by the observing schedule and
%conditions? In broad terms, how would we expect the observing strategy
%to be optimized for this science?


% --------------------------------------------------------------------

\subsection{Metrics}
\label{sec:keyword:metrics}

Quantifying the response via MAF metrics: definition of the metrics,
and any derived overall figure of merit.


% --------------------------------------------------------------------

\subsection{OpSim Analysis}
\label{sec:keyword:analysis}

OpSim analysis: how good would the default observing strategy be, at
the time of writing for this science project?


% --------------------------------------------------------------------

\subsection{Discussion}
\label{sec:keyword:discussion}

Discussion: what risks have been identified? What suggestions could be
made to improve this science project's figure of merit, and mitigate
the identified risks?


% ====================================================================

\navigationbar

% ====================================================================
%+
% SECTION:
%    section-name.tex  % eg lenstimedelays.tex
%
% CHAPTER:
%    chapter.tex  % eg cosmology.tex
%
% ELEVATOR PITCH:
%    Explain in a few sentences what the relevant discovery or
%    measurement is going to be discussed, and what will be important
%    about it. This is for the browsing reader to get a quick feel
%    for what this section is about.
%
% COMMENTS:
%
%
% BUGS:
%
%
% AUTHORS:
%    Phil Marshall (@drphilmarshall)  - put your name and GitHub username here!
%-
% ====================================================================

\section{Dust in the Milky Way}
\def\secname{MW_Dust}\label{sec:\secname} % For example, replace "keyword" with "lenstimedelays"

\noindent{\it Author Name(s)} % (Writing team)

% This individual section will need to describe the particular
% discoveries and measurements that are being targeted in this section's
% science case. It will be helpful to think of a ``science case" as a
% ``science project" that the authors {\it actually plan to do}. Then,
% the sections can follow the tried and tested format of an observing
% proposal: a brief description of the investigation, with references,
% followed by a technical feasibility piece. This latter part will need
% to be quantified using the MAF framework, via a set of metrics that
% need to be computed for any given observing strategy to quantify its
% impact on the described science case. Ideally, these metrics would be
% combined in a well-motivated figure of merit. The section can conclude
% with a discussion of any risks that have been identified, and how
% these could be mitigated.

{\it Peregrine's contribution appears to be missing here... was this overwritten in a commit since August?}\\

A short preamble goes here. What's the context for this science
project? Where does it fit in the big picture?

% --------------------------------------------------------------------

\subsection{Target measurements and discoveries}
\label{sec:keyword:targets}

Describe the discoveries and measurements you want to make.

Now, describe their response to the observing strategy. Qualitatively,
how will the science project be affected by the observing schedule and
conditions? In broad terms, how would we expect the observing strategy
to be optimized for this science?


% --------------------------------------------------------------------

\subsection{Metrics}
\label{sec:keyword:metrics}

Quantifying the response via MAF metrics: definition of the metrics,
and any derived overall figure of merit.

{\bf Metric 1: Uncertainty and bias in $E(B-V)$~estimates as a
  function of location on-sky.} Dependencies:

\begin{itemize}
  \item Stellar population throughout the survey (e.g. Knut / Peter developments; TRILEGAL?);
    \item Dust map throughout the survey region;
    \item Scale photometric error predictions for each band from program requirements per exposure;
      \item Produce formal estimate on the error in extinction and reddening as a function of position on-sky within the survey.
\end{itemize}


% --------------------------------------------------------------------

\subsection{OpSim Analysis}
\label{sec:keyword:analysis}

OpSim analysis: how good would the default observing strategy be, at
the time of writing for this science project?


% --------------------------------------------------------------------

\subsection{Discussion}
\label{sec:keyword:discussion}

Discussion: what risks have been identified? What suggestions could be
made to improve this science project's figure of merit, and mitigate
the identified risks?


% ====================================================================

\navigationbar

% ====================================================================
%+
% SECTION:
%    section-name.tex  % eg lenstimedelays.tex
%
% CHAPTER:
%    chapter.tex  % eg cosmology.tex
%
% ELEVATOR PITCH:
%    Explain in a few sentences what the relevant discovery or
%    measurement is going to be discussed, and what will be important
%    about it. This is for the browsing reader to get a quick feel
%    for what this section is about.
%
% COMMENTS:
%
%
% BUGS:
%
%
% AUTHORS:
%    Phil Marshall (@drphilmarshall)  - put your name and GitHub username here!
%-
% ====================================================================

\section{Microlensing in the Milky Way with LSST}
\def\secname{MW_microlensing}\label{sec:\secname} % For example, replace "keyword" with "lenstimedelays"

\noindent{\it Author Name(s)} % (Writing team)

% This individual section will need to describe the particular
% discoveries and measurements that are being targeted in this section's
% science case. It will be helpful to think of a ``science case" as a
% ``science project" that the authors {\it actually plan to do}. Then,
% the sections can follow the tried and tested format of an observing
% proposal: a brief description of the investigation, with references,
% followed by a technical feasibility piece. This latter part will need
% to be quantified using the MAF framework, via a set of metrics that
% need to be computed for any given observing strategy to quantify its
% impact on the described science case. Ideally, these metrics would be
% combined in a well-motivated figure of merit. The section can conclude
% with a discussion of any risks that have been identified, and how
% these could be mitigated.

%{\bf Note: these metrics have already been worked out in some detail
%  by Gould (arXiv). We do not expect to re-invent that wheel here.}

%A short preamble goes here. What's the context for this science
%project? Where does it fit in the big picture?

Gould (2013) argues assertively for inclusion of microlensing in the
Galactic plane as part of LSST core science; see also Section 4.8.

%Low-mass objects 


% --------------------------------------------------------------------

\subsection{Target measurements and discoveries}
\label{sec:keyword:targets}

Describe the discoveries and measurements you want to make.

Now, describe their response to the observing strategy. Qualitatively,
how will the science project be affected by the observing schedule and
conditions? In broad terms, how would we expect the observing strategy
to be optimized for this science?


% --------------------------------------------------------------------

\subsection{Metrics}
\label{sec:keyword:metrics}

Quantifying the response via MAF metrics: definition of the metrics,
and any derived overall figure of merit.

{\bf Metric 1: fraction of low-mass objects correctly triggered based
  on LSST observational timing.} Dependencies:
\begin{itemize}
  \item How many observations are required for an accurate trigger for followup? (E.g. could simply develop the ``Triples'' of Lund et al. (2015) towards higher number of repeat visits?) 
\item What is the minimum coverage of a microlensing event lightcurve before it is classified as detected and triggered for followup? 
\end{itemize}

{\bf Metric 2: Error on mass function of microlensing objects as a population.} Dependencies in addition to Metric 1, all as spatial map:
\begin{itemize}
  \item Capability to simulate microlens population under assumptions on the mass function
  \item Impact of sampling the mass function on its determination (e.g. if strategy X preferentially misses the short end of the timescale distribution, what does this do to the mass function determination?)
\end{iemize}



% --------------------------------------------------------------------

\subsection{OpSim Analysis}
\label{sec:keyword:analysis}

OpSim analysis: how good would the default observing strategy be, at
the time of writing for this science project?


% --------------------------------------------------------------------

\subsection{Discussion}
\label{sec:keyword:discussion}

Discussion: what risks have been identified? What suggestions could be
made to improve this science project's figure of merit, and mitigate
the identified risks?


% ====================================================================

\navigationbar


%% ====================================================================
%+
% SECTION:
%    section-name.tex  % eg lenstimedelays.tex
%
% CHAPTER:
%    chapter.tex  % eg cosmology.tex
%
% ELEVATOR PITCH:
%    Explain in a few sentences what the relevant discovery or
%    measurement is going to be discussed, and what will be important
%    about it. This is for the browsing reader to get a quick feel
%    for what this section is about.
%
% COMMENTS:
%
%
% BUGS:
%
%
% AUTHORS:
%    Phil Marshall (@drphilmarshall)  - put your name and GitHub username here!
%    Will Clarkson (@willclarkson)
%
%-
% ====================================================================

\section{The Solar Neighborhood}
\def\secname{solarneighborhood}\label{sec:\secname} % For example, replace "keyword" with "lenstimedelays"

\noindent{\it Author Name(s)} % (Writing team)

% This individual section will need to describe the particular
% discoveries and measurements that are being targeted in this section's
% science case. It will be helpful to think of a ``science case" as a
% ``science project" that the authors {\it actually plan to do}. Then,
% the sections can follow the tried and tested format of an observing
% proposal: a brief description of the investigation, with references,
% followed by a technical feasibility piece. This latter part will need
% to be quantified using the MAF framework, via a set of metrics that
% need to be computed for any given observing strategy to quantify its
% impact on the described science case. Ideally, these metrics would be
% combined in a well-motivated figure of merit. The section can conclude
% with a discussion of any risks that have been identified, and how
% these could be mitigated.

A short preamble goes here. What's the context for this science
project? Where does it fit in the big picture?
The science projec

% --------------------------------------------------------------------

\subsection{Target measurements and discoveries}
\label{sec:keyword:targets}

Generally the cases in this science area require deep photometry,
particularly (though not exclusively) at red bandpasses, and
sufficient per-image precision and distribution of observing times in
at least one filter to measure parallax and/or proper motion. The
main-survey area is not sufficient for this science area: dedicated
observations are required for all science cases to fill in angular
coverage (apart from photometric variability of cool dwarfs).

Three representative science cases identified (more are in the Phase I
Science Roadmap):
\begin{itemize}

\item Volume-complete astrometric sample of extended Solar Neighborhood: requires parallax and proper motion, deep multi-filter photometry to classify objects.; 
\item Discovery of ultracold brown dwarfs (late-T and Y); requires very deep photometry in (i,z,Y) 

\item Photometric variability of cool dwarfs; requires sufficient time baseline sampling to chart variability; photometry at other bands to classify object. 

\end{itemize}

%Describe the discoveries and measurements you want to make.

%Now, describe their response to the observing strategy. Qualitatively,
%how will the science project be affected by the observing schedule and
%conditions? In broad terms, how would we expect the observing strategy
%to be optimized for this science?


% --------------------------------------------------------------------

\subsection{Metrics}
\label{sec:keyword:metrics}

Organized by broad science case within this area:
\begin{itemize}
\item Volume-complete astrometric sample of extended Solar Neighborhood:
  \begin{itemize}
  \item Bias and scatter in distance distribution of extended Solar Neighborhood objects;
  \item Bias and scatter in individual source $T_{eff}, \log(g)$~as a
    function of spectral type;
  \item Bias and scatter in population constrants (e.g. mass function slope) for candidate mass functions of local populations;
    \item Fraction of local-volume population recovered {\it and} correctly identified (e.g. spectral subtype error?).
  \end{itemize} 

\item Discovery of ultracold brown dwarfs (late-T/Y):
  \begin{itemize}
  \item Bias and scatter in individual source $T_{eff}, \log(g)$~as a
    function of spectral type;
    \item Bias and scatter in the brown dwarf/exoplanet (?) mass function parameters.
    \item Fraction of objects within $X$~pc from the Sun recovered {\it and} correctly identified (e.g. spectral subtype error?).
  \end{itemize}


\item Photometric variability of cool dwarfs
\begin{itemize}
  \item {\it (Note: hierarchical metric?) E.g. classification of X\% of starspot variability in local volume, which calls Variability Metric $Y$ as a method.}
\end{itemize}

\end{itemize}

%Quantifying the response via MAF metrics: definition of the metrics,
%and any derived overall figure of merit.

% --------------------------------------------------------------------

\subsection{OpSim Analysis}
\label{sec:keyword:analysis}

OpSim analysis: how good would the default observing strategy be, at
the time of writing for this science project?


% --------------------------------------------------------------------

\subsection{Discussion}
\label{sec:keyword:discussion}

One useful output of MAF for Solar Neighborhood work, will be the
extent to which the small part of the sky in front of crowded regions
in this general science area will make an impact. 

Comparison of the science metrics for cases when Galactic Plane
observations are / are not included as part of the OpSim run, will
directly address the degree to which LSST should consider Solar
Neighborhood science in its galactic plane strategy.

%Discussion: what risks have been identified? What suggestions could be
%made to improve this science project's figure of merit, and mitigate
%the identified risks?


% ====================================================================

\navigationbar

%% ====================================================================
%+
% SECTION:
%    section-name.tex  % eg lenstimedelays.tex
%
% CHAPTER:
%    chapter.tex  % eg cosmology.tex
%
% ELEVATOR PITCH:
%    Explain in a few sentences what the relevant discovery or
%    measurement is going to be discussed, and what will be important
%    about it. This is for the browsing reader to get a quick feel
%    for what this section is about.
%
% COMMENTS:
%
%
% BUGS:
%
%
% AUTHORS:
%    Phil Marshall (@drphilmarshall)  - put your name and GitHub username here!
%    Will Clarkson (@willclarkson)
%
%-
% ====================================================================

\section{Star Clusters}
\def\secname{starclusters}\label{sec:\secname} % For example, replace "keyword" with "lenstimedelays"

\noindent{\it Author Name(s)} % (Writing team)

% This individual section will need to describe the particular
% discoveries and measurements that are being targeted in this section's
% science case. It will be helpful to think of a ``science case" as a
% ``science project" that the authors {\it actually plan to do}. Then,
% the sections can follow the tried and tested format of an observing
% proposal: a brief description of the investigation, with references,
% followed by a technical feasibility piece. This latter part will need
% to be quantified using the MAF framework, via a set of metrics that
% need to be computed for any given observing strategy to quantify its
% impact on the described science case. Ideally, these metrics would be
% combined in a well-motivated figure of merit. The section can conclude
% with a discussion of any risks that have been identified, and how
% these could be mitigated.

A short preamble goes here. What's the context for this science
project? Where does it fit in the big picture?

[Star cluster leads]

%Two broad science areas {\bf [more?]}: (i) Better understanding of
%stellar populations within clusters; (ii) star clusters as tracers of
%star formation in the Galaxy, including donation to the Plane by
%dissolution

% --------------------------------------------------------------------

\subsection{Target measurements and discoveries}
\label{sec:keyword:targets}


\begin{itemize}

\item Formation history and evolution of the Milky Way as traced by star clusters

\item Better understanding of star cluster stellar populations; stellar mass function, metallicity, ages

\end{itemize}

%Describe the discoveries and measurements you want to make.

%Now, describe their response to the observing strategy. Qualitatively,
%how will the science project be affected by the observing schedule and
%conditions? In broad terms, how would we expect the observing strategy
%to be optimized for this science?


% --------------------------------------------------------------------

\subsection{Metrics}
\label{sec:keyword:metrics}

Organized by broad science case within this area:
\begin{itemize}
\item Milky Way formation and evolution as traced by star clusters (focus below on tidal dissolution $\rightarrow$~contribution of stars to the Disk by clusters):
  \begin{itemize}
    \item Radius $R$ from cluster center at which LSST becomes unable to kinematically identify cluster members (figure might be: [x,y,z] = [cluster $M_v$, concentration parameter, $R$]; rerun for different locations in the Galaxy?
      \item Sensitivity (quantified as surface brightness limit?) to tidal streams between clusters? Perhaps go broader, e.g. mass ratio or $M_v$~for tidally interacting clusters for which LSST becomes insensitive to existence of tidal tail?
        \item LSST's sensitivity to azimuthal asymmetries in the light (or mass) profile of clusters ``near'' the plane
  \end{itemize} 

\item Stellar populations within clusters; mass function, metallicity, ages
  \begin{itemize}
    \item Radius of avoidance for LSST per-filter for a catalog of
      known clusters (in-plane and out)
      \item Distance limit at which LSST becomes insensitive to stars above mass $m_{lower}$ 
  \end{itemize}

\end{itemize}

(Metrics note: all of the above are likely to be hierarchical, calling
some parameterization of photometric sensitivity vs crowding as a
method.)

Need some sort of metric for the science return on the ensemble of
globular clusters observed by LSST. If LSST doesn't go deeper than X
at latitude $|b| < Y$, what is the impact on the science?

%Quantifying the response via MAF metrics: definition of the metrics,
%and any derived overall figure of merit.

% --------------------------------------------------------------------

\subsection{OpSim Analysis}
\label{sec:keyword:analysis}

OpSim analysis: how good would the default observing strategy be, at
the time of writing for this science project?


% --------------------------------------------------------------------

\subsection{Discussion}
\label{sec:keyword:discussion}

One useful output of MAF for Solar Neighborhood work, will be the
extent to which the small part of the sky in front of crowded regions
in this general science area will make an impact. 

Comparison of the science metrics for cases when Galactic Plane
observations are / are not included as part of the OpSim run, will
directly address the degree to which LSST should consider Solar
Neighborhood science in its galactic plane strategy.

%Discussion: what risks have been identified? What suggestions could be
%made to improve this science project's figure of merit, and mitigate
%the identified risks?


% ====================================================================

\navigationbar

%% ====================================================================
%+
% SECTION:
%    section-name.tex  % eg lenstimedelays.tex
%
% CHAPTER:
%    chapter.tex  % eg cosmology.tex
%
% ELEVATOR PITCH:
%    Explain in a few sentences what the relevant discovery or
%    measurement is going to be discussed, and what will be important
%    about it. This is for the browsing reader to get a quick feel
%    for what this section is about.
%
% COMMENTS:
%
%
% BUGS:
%
%
% AUTHORS:
%    Phil Marshall (@drphilmarshall)  - put your name and GitHub username here!
%    Will Clarkson (@willclarkson)
%
%-
% ====================================================================

\section{Halo Structure}
\def\secname{halostructure}\label{sec:\secname} % For example, replace "keyword" with "lenstimedelays"

\noindent{\it Author Name(s)} % (Writing team)

% This individual section will need to describe the particular
% discoveries and measurements that are being targeted in this section's
% science case. It will be helpful to think of a ``science case" as a
% ``science project" that the authors {\it actually plan to do}. Then,
% the sections can follow the tried and tested format of an observing
% proposal: a brief description of the investigation, with references,
% followed by a technical feasibility piece. This latter part will need
% to be quantified using the MAF framework, via a set of metrics that
% need to be computed for any given observing strategy to quantify its
% impact on the described science case. Ideally, these metrics would be
% combined in a well-motivated figure of merit. The section can conclude
% with a discussion of any risks that have been identified, and how
% these could be mitigated.

A short preamble goes here. What's the context for this science
project? Where does it fit in the big picture?

%Two broad science areas {\bf [more?]}: (i) Better understanding of
%stellar populations within clusters; (ii) star clusters as tracers of
%star formation in the Galaxy, including donation to the Plane by
%dissolution

% --------------------------------------------------------------------

\subsection{Target measurements and discoveries}
\label{sec:keyword:targets}

Summary: with one exception (stellar streams and overdensities), most
halo cases appear to be adequately met by the main survey. A
representative sample of halo cases are included here for
completeness. Main requirements: brightness and color precision;
proper motion (some cases); variability (some cases).


===

Describe the discoveries and measurements you want to make.

Now, describe their response to the observing strategy. Qualitatively,
how will the science project be affected by the observing schedule and
conditions? In broad terms, how would we expect the observing strategy
to be optimized for this science?


% --------------------------------------------------------------------

\subsection{Metrics}
\label{sec:keyword:metrics}

Quantifying the response via MAF metrics: definition of the metrics,
and any derived overall figure of merit.

% --------------------------------------------------------------------

\subsection{OpSim Analysis}
\label{sec:keyword:analysis}

OpSim analysis: how good would the default observing strategy be, at
the time of writing for this science project?


% --------------------------------------------------------------------

\subsection{Discussion}
\label{sec:keyword:discussion}

Discussion: what risks have been identified? What suggestions could be
made to improve this science project's figure of merit, and mitigate
the identified risks?


% ====================================================================

\navigationbar

%% ====================================================================
%+
% SECTION:
%    section-name.tex  % eg lenstimedelays.tex
%
% CHAPTER:
%    chapter.tex  % eg cosmology.tex
%
% ELEVATOR PITCH:
%    Explain in a few sentences what the relevant discovery or
%    measurement is going to be discussed, and what will be important
%    about it. This is for the browsing reader to get a quick feel
%    for what this section is about.
%
% COMMENTS:
%
%
% BUGS:
%
%
% AUTHORS:
%    Phil Marshall (@drphilmarshall)  - put your name and GitHub username here!
%    Will Clarkson (@willclarkson)
%
%-
% ====================================================================

\section{The Local Volume}
\def\secname{localvolume}\label{sec:\secname} % For example, replace "keyword" with "lenstimedelays"

\noindent{\it Author Name(s)} % (Writing team)

% This individual section will need to describe the particular
% discoveries and measurements that are being targeted in this section's
% science case. It will be helpful to think of a ``science case" as a
% ``science project" that the authors {\it actually plan to do}. Then,
% the sections can follow the tried and tested format of an observing
% proposal: a brief description of the investigation, with references,
% followed by a technical feasibility piece. This latter part will need
% to be quantified using the MAF framework, via a set of metrics that
% need to be computed for any given observing strategy to quantify its
% impact on the described science case. Ideally, these metrics would be
% combined in a well-motivated figure of merit. The section can conclude
% with a discussion of any risks that have been identified, and how
% these could be mitigated.

A short preamble goes here. What's the context for this science
project? Where does it fit in the big picture?

[Star cluster leads]

%Two broad science areas {\bf [more?]}: (i) Better understanding of
%stellar populations within clusters; (ii) star clusters as tracers of
%star formation in the Galaxy, including donation to the Plane by
%dissolution

% --------------------------------------------------------------------

\subsection{Target measurements and discoveries}
\label{sec:keyword:targets}

Describe the discoveries and measurements you want to make.

Now, describe their response to the observing strategy. Qualitatively,
how will the science project be affected by the observing schedule and
conditions? In broad terms, how would we expect the observing strategy
to be optimized for this science?


% --------------------------------------------------------------------

\subsection{Metrics}
\label{sec:keyword:metrics}

Quantifying the response via MAF metrics: definition of the metrics,
and any derived overall figure of merit.

% --------------------------------------------------------------------

\subsection{OpSim Analysis}
\label{sec:keyword:analysis}

OpSim analysis: how good would the default observing strategy be, at
the time of writing for this science project?


% --------------------------------------------------------------------

\subsection{Discussion}
\label{sec:keyword:discussion}

Discussion: what risks have been identified? What suggestions could be
made to improve this science project's figure of merit, and mitigate
the identified risks?


% ====================================================================

\navigationbar


\section{Milky Way Metrics: future work}
\def\secname{MW_future_work}\label{sec:\secname}

Analysis of precursor data using the LSST stack is expected on a
relatively short timescale (perhaps during 2016-2017)? The catalogs
thus produced, and their source images, will provide the opportunity
for very fine-grained information to be folded in to metrics that
relate to crowded regions.


\navigationbar

% 2015-08-22: previous versions kept as comments for now, to help copy
% up old material in a non github-expert way in future

%\subsection{The Galactic Bulge}

%Summary: Nearly entirely in LSST’s classical region of avoidance, thus
%dedicated observations will be needed. Want to optimize for photometry
%and astrometry, as well as variability on a timescale of hours or
%longer (for RR Lyrae and other tracers of structure). Quite sensitive
%to crowding. At the longer timescale, microlensing with a wide-area
%facility like LSST could be transformative if affordable.

%NOT required for optimization: parallax

%(Note: the final science case, microlensing, might be a good candidate
%for a deep-drilling-like survey, e.g. pick a 2$\times$2 set of LSST
%pointings and monitor those with microlensing-friendly cadence in two
%filters (to help weed out false positives). This is one science case
%that should be revisited anyway.)


%\subsubsection{Bulge structure and stellar populations}
%\vspace{-2mm}
%\begin{itemize}
%\item {\bf Observing requirement:} Multi-color photometry in all bands to disen%tangle constituent bulge populations (required); sensitivity to RR Lyrae for di%stance and also extinction mapping (required); relative proper motion sensitivi%ty for kinematic population separation (preferred; requires proper motion preci%sion at the 0.5-1mas/yr level).
%\vspace{-2mm}

%\item {\bf Strategy requirement:} \underline{\it Maximum} individual exposure t%ime in these crowded fields; minimum exposure time for u-band sensitivity; cade%nce sufficient for proper motions; cadence sufficient for sensitivity to variab%les on $\sim$hour-long timescales.
%\vspace{-2mm}

%\item {\bf Discovery space for LSST:} Populations down to the main sequence tur%n-off (most fields); balance of populations across the entire structure; discov%ery of new RR Lyrae and improvement of extinction map. Gaia cannot make precisi%on measurements down to a couple of magnitudes or so above the turn-off in thes%e fields {\it (do we have more quantitative information than this yet?)}.
%\vspace{-2mm}

%\item {\bf New observations required?} Yes - the bulge is outside the
%  main LSST survey area. In addition, observations of disk-calibration
%  fields will be needed for statistical subtraction of the
%  foreground. Short exposures will also be needed to constrain nearby
%  bright objects or at least characterize their effect on the deep
%  exposures.
%\vspace{2mm}
%\end{itemize}

%\subsubsection{Stellar kinematics in the Bulge and foreground}
%\vspace{-2mm}
%\begin{itemize}
%\item {\bf Observing requirement:} Proper motion sensitivity better than 0.5 ma%s/yr, in at least one band.
%\vspace{-2mm}

%\item {\bf Strategy requirement:} \underline{\it Maximum} individual exposure time in these crowded fields; cadence distributed to maximize proper motion precision.
%\vspace{-2mm}

%\item {\bf Discovery space for LSST:} Proper motions both internally to LSST and externally to earlier epochs from previous campaigns. Gaia cannot make precision measurements down to a couple of magnitudes or so above the turn-off.
%\vspace{-2mm}

%\item {\bf New observations required?} Yes - the bulge is outside the main LSST survey area. Short exposures will also be needed to constrain nearby bright objects or at least characterize their effect on the deep exposures.
%\vspace{-2mm}
%\end{itemize}

%\subsubsection{Low-mass microlens events towards the Bulge}

%\begin{itemize}
%\item {\bf Observing requirement:} Main-survey-like monitoring in a subset of f%ilters.
%\vspace{-2mm}

%\item {\bf Strategy requirement:} \underline{\it Maximum} individual exposure time in these crowded fields (likely different from the previous two science cases due to different analysis techniques); preferred filter choice for monitoring; cadence for sensitivity to microlensing events.
%\vspace{-2mm}

%\item {\bf Discovery space for LSST:} Detection of microlensing events at the low-mass end, including free-floating planets, at levels inaccessible to smaller-aperture trigger surveys. These objects would later be followed up by dedicated observations with other facilities.
%\vspace{-2mm}

%\item {\bf New observations required?} Yes - the bulge is outside the main LSST survey area. Short exposures will also be needed to constrain nearby bright objects or at least characterize their effect on the deep exposures.

%{\it Note: this might be a good candidate for a deep-drilling-like survey, e.g. pick a 2$\times$2 set of LSST pointings and monitor those with microlensing-friendly cadence in two filters (to help weed out false positives).}

%\vspace{-2mm}
%\end{itemize}

%\subsection{The Milky Way Disk}

%Summary: precise colors and magnitudes, with the ability to
%disentangle populations in crowded regions. Sufficient cadence for
%variability down to minutes-hours variations.

%NOT required for optimization: proper motion, parallax.

%\subsubsection{Thin disk/thick disk structure and stellar populations}

%\begin{itemize}
%\item {\bf Observing requirement:} Precise magnitudes and colors; proper motions for reduced proper motion analysis. {\it (To what level of precision?)} Sensitivity in variability to RR Lyrae,  $\delta$ Scuti variables, eclipsing binaries.
%\vspace{-2mm}

%\item {\bf Strategy requirement:} \underline{\it Maximum} individual exposure time to avoid crowding per exposure; minimum total exposure time. Cadence sufficient for proper motions; cadence sufficient for variability down to a timescale of minutes-hours.
%\vspace{-2mm}

%\item {\bf Discovery space for LSST:} Regions too crowded for gaia
%\vspace{-2mm}


%\item {\bf New observations required?} Yes - these are observations of the Galactic Plane. Short exposures will also be needed to constrain nearby bright objects or at least characterize their effect on the deep exposures.
%\vspace{-2mm}
%\end{itemize}

%\subsubsection{Star formation in the Galactic Disk}

% New content moved across to MW_SFH.tex

%Summary: this important topic does not seem to have been developed in
%previous versions of the LSST science book or Ivezic et
%al. (2008). LSST gives the opportunity to survey extensive areas
%around star formation regions in the Southern hemisphere. Among
%others, it would allow to study the Initial Mass Function down to the
%sub-stellar limit across different environments. Young stars are
%efficiently identified by their variability.

%In order to assess the ability of LSST to 1) identify and 2) classify YSO we need to quantify the variability timescales and amplitudes of both Class I/II (stars with disks) and Class III (WTTS). Inclusion of eruptive variables (FUor/Exor) is appropriate as well - see section 8.10.2 in the Science Book.

%In brief, WTTS are quasi-periodic with amplitudes of 0.1 to 0.3 mag and periods 1 to ~15 days - so comparable to gamma Dor stars (see Figure 8.17 in the SB). Given the temporal evolution of cool spots, a period recovery analysis such as shown for RRL stars (se FIgure 8.20 in the SB) is likely difficult. The embedded systems and CTTS are irregular variables but shown have distinctive colors due to extinction + UB/blue excess arising from to accretion shocks.


%NOT required for optimization: parallax, relative proper motion.

%\begin{itemize}
%\item {\bf Observing requirement:} Precise magnitudes and colors, appropriate cadence for T Tauri variability (days).

%\vspace{-2mm}

%\item {\bf Strategy requirement:} filter set for population constraints; redder bands (z, Y) important for the lowest mass stars; u band for accretion rates
%\vspace{-2mm}

%\item {\bf Discovery space for LSST:} Only optical survey in the galactic plane. It will produce an unbiased map of the young stellar populations in the Southern Hemisphere. Possibility of early alerts for outbursts of young stars.
%\vspace{-2mm}

%\item {\bf New observations required?} Yes, most regions are within the galactic plane (outside the LSST main survey area). Constraints in spatial coverage can be done by defining the areas of star formation.
%\vspace{-2mm}
%\end{itemize}

%\subsubsection{Spiral structure in the Milky Way disk}

%{\it Note: science case could use some development. This is based on LSST Science Book section 7.3.2.}

%\begin{itemize}
%\item {\bf Observing requirement:} Colors, photometry, proper motions
%\vspace{-2mm}

%\item {\bf Strategy requirement:} Minimum filter-set; cadence sufficient for proper motion; exposure time and strategy optimized to crowding in the preferred fields (e.g. $l \sim 270$).
%\vspace{-2mm}

%\item {\bf Discovery space for LSST:} Large, coherent structures in phase space that would be difficult for smaller-etendue surveys to efficiently probe. Regions too crowded for gaia.
%\vspace{-2mm}

%\item {\bf New observations required?} Yes - the galactic plane is not part of the LSST main survey.
%\vspace{-2mm}
%\end{itemize}


%\subsection{Dust throughout the Milky Way}

%Summary: deep \{ugriz\} observations required; usefulness appears to depend mainly on the depth achieved in each filter.

%NOT required for optimization: variability, parallax, proper motion,
%y-band photometry [? - not according to the LSST Science book 7.5]

%\subsubsection{Spatial distribution of dust}

%\begin{itemize}
%\item {\bf Observing requirement:} As deep as possible in \{ugrizy\} to allow reddening-free indices to be constructed for stars, in order for the intrinsic and true colors to be compared to estimate reddening.
%\vspace{-2mm}

%\item {\bf Strategy requirement:} Minimum exposure-time accumulated in each filter; minimum filter-set {\it (Is y-band needed for this? Presumably would help, but doesn't seem to be mentioned in the Science book 7.5.)}.
%\vspace{-2mm}

%\item {\bf Discovery space for LSST:} 3D dust maps out to much greater distance than previously possible (e.g. with SDSS)
%\vspace{-2mm}

%\item {\bf New observations required?} Yes, but only for regions in the Galactic plane.
%\vspace{-2mm}
%\end{itemize}


%\subsubsection{Variation in extinction laws}

%\begin{itemize}
%\item {\bf Observing requirement:} As deep as possible in \{ugrizy\} to estimate changes in reddening vector as a function of position and depth.
%\vspace{-2mm}

%\item {\bf Strategy requirement:} Minimum exposure-time accumulated in each filter; minimum filter-set; 2\% photometric accuracy in \{ugriz\}. {(Again, is y required?)}
%\vspace{-2mm}

%\item {\bf Discovery space for LSST:} F-turnoff stars with $g > 19$~(fainter than gaia will measure).
%\vspace{-2mm}

%\item {\bf New observations required?} Yes, but only for regions in the Galactic plane.
%\vspace{-2mm}
%\end{itemize}

%\subsection{The Halo}

%Summary: with one exception (stellar streams and overdensities), most
%halo cases appear to be adequately met by the main survey. A
%representative sample of halo cases are included here for
%completeness. Main requirements: brightness and color precision;
%proper motion (some cases); variability (some cases).

%NOT required for optimization: parallax

%\subsubsection{Halo stellar streams and overdensities}

%\begin{itemize}
%\item {\bf Observing requirement:} Deep photometry, with sufficient color precision to identify main sequence objects. Variability sufficient to discern RR Lyrae.
%\vspace{-2mm}

%\item {\bf Strategy requirement:} Minimum filter-set required; minimum exposure time in these filters; variability sufficient for RR Lyrae.
%\vspace{-2mm}

%\item {\bf Discovery space for LSST:} Objects too faint for gaia or PanSTARRS (seems to be most of the sample); structures with a very large extent on the sky; structures close to the Galactic Plane.
%\vspace{-2mm}

%\item {\bf New observations required?} Not for most of the sky, since the main survey has been shown to be sufficient for RR Lyrae. However, overdensities close to the Galactic Plane (like the Monoceros Ring) may be located outside the main survey, and then require dedicated observations. These observations would then need to be optimized carefully to achieve sufficient coverage for the desired tracers.
%\vspace{-2mm}
%\end{itemize}

%\subsubsection{Halo structure: main sequence stars out to 300 kpc}

%\begin{itemize}
%\item {\bf Observing requirement:} Deep photometry, with sufficient color precision to identify main sequence objects.
%\vspace{-2mm}

%\item {\bf Strategy requirement:} Minimum filter-set required; minimum exposure time in these filters
%\vspace{-2mm}

%\item {\bf Discovery space for LSST:} Objects too faint for gaia or PanSTARRS
%\vspace{-2mm}

%\item {\bf New observations required?} No - baseline survey should be sufficient (e.g. Ivezic et al. 2008 2.1.5).
%\vspace{-2mm}
%\end{itemize}

%\subsubsection{Hypervelocity stars in and in front of the Halo}

%\begin{itemize}
%\item {\bf Observing requirement:} Proper motion precision at the 1 mas/yr level; brightness and color precision sufficient to constrain luminosity class
%\vspace{-2mm}

%\item {\bf Strategy requirement:} Minimum filter-set required; minimum exposure time in these filters; cadence for proper motions
%\vspace{-2mm}

%\item {\bf Discovery space for LSST:} Old main sequence turn-off stars at about 10kpc (r < 20; brighter than this is accessible to gaia)
%\vspace{-2mm}

%\item {\bf New observations required?} No - LSST science book 7.7
%\vspace{-2mm}
%\end{itemize}

%\subsubsection{The most metal-poor stars in the Galaxy}

%\begin{itemize}
%\item {\bf Observing requirement:} Color, brightness precision in the full \{ugrizy\} set for photometric selection of candidate metal-poor stars out to 100kpc from the Galactic Center.
%\vspace{-2mm}

%\item {\bf Strategy requirement:} Minimum filter-set required; minimum exposure time in these filters; cadence for proper motions
%\vspace{-2mm}

%\item {\bf Discovery space for LSST:} A much larger sample of metal-poor stars over a wider area than previously possible.
%\vspace{-2mm}

%\item {\bf New observations required?} No - LSST science book 6.7.
%\vspace{-2mm}
%\end{itemize}

%\navigationbar


% --------------------------------------------------------------------

% Old notes:

%\subsubsection{}

%\begin{itemize}
%\item {\bf Observing requirement:}
%\vspace{-2mm}

%\item {\bf Strategy requirement:}
%\vspace{-2mm}

%\item {\bf Discovery space for LSST:}
%\vspace{-2mm}

%\item {\bf New observations required?}
%\vspace{-2mm}
%\end{itemize}

%\subsubsection{}

%\begin{itemize}
%\item {\bf Observing requirement:}
%\vspace{-2mm}

%\item {\bf Strategy requirement:}
%\vspace{-2mm}

%\item {\bf Discovery space for LSST:}
%\vspace{-2mm}

%\item {\bf New observations required?}
%\vspace{-2mm}
%\end{itemize}
