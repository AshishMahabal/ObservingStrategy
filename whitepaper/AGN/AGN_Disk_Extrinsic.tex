% ====================================================================
%+
% SECTION:
%    AGN_Microlensing.tex
%
% CHAPTER:
%    AGN.tex
%
% ELEVATOR PITCH:
%    Using AGN microlensing to measure the size and structure of
%    accretion disks. Depends on well-sampled multi-filter light curves,
%    and a large sample of detected strongly-lensed AGN.
%
% COMMENTS:
%
%
% BUGS:
%
%
% AUTHORS:
%    Timo Anguita (@tanguita)
%-
% ====================================================================

\section{AGN Size and Structure with Microlensing}
\def\secname{\chpname:microlensing}\label{sec:\secname}

\credit{tanguita}

% This individual section will need to describe the particular
% discoveries and measurements that are being targeted in this section's
% science case. It will be helpful to think of a ``science case" as a
% ``science project" that the authors {\it actually plan to do}. Then,
% the sections can follow the tried and tested format of an observing
% proposal: a brief description of the investigation, with references,
% followed by a technical feasibility piece. This latter part will need
% to be quantified using the MAF framework, via a set of metrics that
% need to be computed for any given observing strategy to quantify its
% impact on the described science case. Ideally, these metrics would be
% combined in a well-motivated figure of merit. The section can conclude
% with a discussion of any risks that have been identified, and how
% these could be mitigated.

% A short preamble goes here. What's the context for this science
% project? Where does it fit in the big picture?

Microlensing due to stars projected on top of individual
gravitationally-lensed quasar images produce additional magnification.
Using the fact that the Einstein radii of stars in lensing galaxies
closely match the scales of different emission regions in
high-redshift AGNs (micro-arcseconds), analyzing microlensing induced
flux variations statistically on individual systems allows us to
measure ``sizes'' of AGN regions.

Assuming a thermal profile for accretion disks, sizes in different
emission wavelengths will be probed and as such, constraints on the
slope of this thermal profile will be placed. Given the sheer number
of lensed systems that LSST is expected to discover ($\sim8000$),
this will allow us to stack systems for better constraints and
hopefully determine the {\it evolution of the size and profile.} Due to the
typical relative velocities of lenses, microlenses, observers (Earth)
and source AGN, the microlensing variation timescales are between
months to a few decades.


% --------------------------------------------------------------------

\subsection{Target measurements and discoveries}
\label{sec:\secname:targets}

% Describe the discoveries and measurements you want to make.

% Now, describe their response to the observing strategy. Qualitatively,
% how will the science project be affected by the observing schedule and
% conditions? In broad terms, how would we expect the observing strategy
% to be optimized for this science?

\new{Our goal is to understand the population of AGN accretion disk sizes
and profiles. We anticipate doing this via a hierarchical model where
these properties are related to each other in some way, perhaps via
power law scaling relations. A very simple version of this is the following...
\newline\newline
So, our targets are the parameters $a$ and $b$, that describe this
simple population. How well will we be able to measure these, for a
given survey strategy? This is what our MAF metric will quantify.
Before we design this, we can predict the likely sensitivities of this
measurement.}


The quasar microlensing optical depth is $\sim1$, so every lensed
quasar should be affected by microlensing at any given point in time.
However, measurable variability can occur on longer timescales.
\citet{MosqueraandKochanek2011} studied all known lensed quasars.
They found the median timescale between high magnification events
(Einstein crossing time scales) in the observed $I$-band is of the
order of $\sim20$~yr (with a distribution between 10 and 40~yr).
However, the source crossing time (duration of a high magnification
event) is $\sim7.3$~months (with a distribution tail up to 3~yr).
This basically means that out of all the lensed quasar {\em images}
(microlensing between images is completely uncorrelated) about half
of them will be quiescent during the 10~yr baseline of LSST. However,
since the typical number of lensed images is either two or four, it
means that, statistically, in every system, one (for doubles) or two
(for quads) high magnification events should be observed in 10~yr of
LSST monitoring.

Note that, the important cadence parameter is the source crossing time,
as it is the length of the event to be as uniformly sampled as
possible. The 7.3 months crossing time is the median for the observed
$i$-band, but this time would be significantly shorter for bluer bands:
for a thermal profile with slope
$\alpha: R_\lambda \propto \lambda^\alpha$ implies source crossing time
$t_{\rm s} \propto \lambda^{1/\alpha} \rightarrow
t_u=t_i \times (\lambda_{\rm u} / \lambda_{\rm i})^{1/\alpha}$. For a
Shakura-Sunyaev slope of $\alpha=0.75$ this would correspond to
$7.3 \times (3600/8140)^{4/3}$ months which is $\approx 2.5$ months in
the $u$-band.

In terms of the cadence, at least three evenly sampled data points per
band within two to three months would be preferred to be able to map
the constraining high magnification event(?), and these would
hopefully be uniformly spaced. Very tight cadence (e.g., DDFs) would
increase the constraints significantly. However, since lensed quasars
are not that common, this smaller area would mean only a few
($\sim80$?) suitable systems monitored in the DDFs.

Regarding the season length, the ``months'' timescale of high
magnification events very likely means that we can/will miss high
magnification events in the season gaps, at least in the bluer bands.

Show stopper: observations spread on timescales larger than 3 months(?).
This would likely miss the high magnification events. In those cases
we could perhaps consider close consecutive photometric bands as
equivalent accretion disk regions, however this would mean weaker
constraints on the thermal profile.
%
Important Note: all this science needs to be done on lensed quasars
with measured or very short time delays to remove the intrinsic
variability signal, which might significantly reduce the sample.

{\bf Microlensing Aided Reverberation Mapping:} Given that
microlensing mostly affects continuum emission rather than BELR line
emission, microlensing may enable disentangling the BELR line plus the
continuum emission in single photometric bands, allowing the use of
single broad band PRM measurements \citep{SluseandTewes2014}. As with
the two-band PRM method discussed above, the denser (and the longer)
the sampling, the more accurate are the constraints that can be
obtained for the time delays.

% --------------------------------------------------------------------

\subsection{Metrics}
\label{sec:\secname:metrics}

Quantifying the response via MAF metrics: definition of the metrics,
and any derived overall figure of merit.

Need to compute the dispersion in the time gap
between visits in the same band, across the sky, in order to assess
the fraction of microlensing events that might be missed (on top of
seasonal gaps).

% microlensing - convolve microlensing timescales for QSOs we already know
% about. how many of the high magnification events do we get? How bright?
% @tanguita

% --------------------------------------------------------------------

\subsection{OpSim Analysis}
\label{sec:\secname:analysis}

OpSim analysis: how good would the default observing strategy be, at
the time of writing for this science project?


% --------------------------------------------------------------------

\subsection{Discussion}
\label{sec:\secname:discussion}

Discussion: what risks have been identified? What suggestions could be
made to improve this science project's figure of merit, and mitigate
the identified risks?


% ====================================================================

\navigationbar
