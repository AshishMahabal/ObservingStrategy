% ====================================================================
%+
% SECTION:
%    section-name.tex  % eg lenstimedelays.tex
%
% CHAPTER:
%    chapter.tex  % eg cosmology.tex
%
% ELEVATOR PITCH:
%    Explain in a few sentences what the relevant discovery or
%    measurement is going to be discussed, and what will be important
%    about it. This is for the browsing reader to get a quick feel
%    for what this section is about.
%
% COMMENTS:
%
%
% BUGS:
%
%
% AUTHORS:
%    Phil Marshall (@drphilmarshall)  - put your name and GitHub username here!
%-
% ====================================================================

\section{AGN Variability}
\def\secname{\chpname:variability}\label{sec:\secname}

\credit{ohadshemmer},
{\it and others to follow}

% This individual section will need to describe the particular
% discoveries and measurements that are being targeted in this section's
% science case. It will be helpful to think of a ``science case" as a
% ``science project" that the authors {\it actually plan to do}. Then,
% the sections can follow the tried and tested format of an observing
% proposal: a brief description of the investigation, with references,
% followed by a technical feasibility piece. This latter part will need
% to be quantified using the MAF framework, via a set of metrics that
% need to be computed for any given observing strategy to quantify its
% impact on the described science case. Ideally, these metrics would be
% combined in a well-motivated figure of merit. The section can conclude
% with a discussion of any risks that have been identified, and how
% these could be mitigated.

% A short preamble goes here. What's the context for this science
% project? Where does it fit in the big picture?

A variety of AGN variability studies will be
enabled by LSST. These are intended to probe the physical properties
of the unresolved inner regions of the central engine. Relations will
be sought between variability amplitude and timescale vs. $L$, $z$,
$\lambda_{\rm eff}$, color, multiwavelength and spectroscopic
properties, if available. The LSST sampling is expected to provide
high-quality power spectral density (PSD) functions for a large number
of AGNs; these can be used to constrain the SMBH mass and accretion
rate/mode. Furthermore, LSST AGNs exhibiting excess variability over
that expected from their luminosities will be further scrutinized as
candidates for lensed systems having unresolved images with the excess
(extrinsic) variability being attributed mainly to microlensing.

Potentially periodic AGN variability, leading to tentative discoveries
of binary SMBHs (e.g., \citet{GrahamEtal2015}), may also be
measurable.  Over the ten-year survey, LSST will be sensitive to
periods of a few days up to $\sim3$~yr.



% --------------------------------------------------------------------

\subsection{Target measurements and discoveries}
\label{sec:\secname:targets}

Describe the discoveries and measurements you want to make.

Now, describe their response to the observing strategy. Qualitatively,
how will the science project be affected by the observing schedule and
conditions? In broad terms, how would we expect the observing strategy
to be optimized for this science?

\new{We focus on the PSD function as a way of characterizing AGN
variability in various ways. What do we expect the AGN population to
look like in PSD parameter space? The hyper-parameters that govern the
relationships between PSD parameters and  AGN and host galaxy
properties are probably of greatest scintific interest.}

% --------------------------------------------------------------------

\subsection{Metrics}
\label{sec:\secname:metrics}

Quantifying the response via MAF metrics: definition of the metrics,
and any derived overall figure of merit.

\new{In lieu of a simulated AGN population, we focus on a few
particular {\it diagnostic} metrics that capture  our likely ability
to measure the PSD across the population. These include: the
uniformity of the sampling pattern in log time lag?}

% QPOs - Will the cadence and duration give you the proper range on the
% power spectrum to detect QPOs with a given mass, spin, and L/Ledd?
% (Bob Wagoner)


% --------------------------------------------------------------------

\subsection{OpSim Analysis}
\label{sec:\secname:analysis}

OpSim analysis: how good would the default observing strategy be, at
the time of writing for this science project?


% --------------------------------------------------------------------

\subsection{Discussion}
\label{sec:\secname:discussion}

Discussion: what risks have been identified? What suggestions could be
made to improve this science project's figure of merit, and mitigate
the identified risks?


% ====================================================================

\navigationbar
