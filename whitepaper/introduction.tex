
\chapter[Introduction]{Introduction}
\def\chpname{intro}\label{chp:\chpname}

\noindent {\it
Phil Marshall, Zeljko Ivezic, Beth Willman, ...
}

The Large Synoptic Survey Telescope (LSST) is a dedicated optical
telescope with an effective aperture of 6.7 meters, currently under
construction on Cerro Pach\'on in the Chilean Andes.  The telescope
and camera will have a huge field of view, 9.6 deg$^2$, and the
\'etendue, i.e., the product of collecting area and field of view will
be significantly larger than any other telescope.  Thus this telescope
is designed for wide-field deep imaging of the sky; its mantra is
``Wide-Deep-Fast'', i.e., the ability to cover large swaths of sky
(``Wide'') to faint magnitudes (``Deep'') in a short amount of time
(``Fast''), allowing to scan the sky repeatedly. 

  The science case for the LSST is based broadly on four science
  themes: 
\begin{itemize} 
\item Dark energy and dark matter (via strong and weak lensing,
  large-scale structure, and supernovae);
\item Exploring the transient and variable universe; 
\item Studying the structure of the Milky Way galaxy and its neighbors
  via resolved stellar populations;
\item An inventory of the Solar System, including Near-Earth Asteroids
  and Potential Hazardous Objects, Main-Belt Asteroids, and
  Kuiper-Belt Objects.
\end{itemize}

These themes, together with {\em many} other science applications, are
described in detail in the
\href{http://lsst.org/scientists/scibook}{LSST Science Book}, produced
by the LSST Project Team and Science collaborations in 2009.  The
present white paper represents an important next step in science
planning beyond the Science Book.  In particular, we now need to
quantify how well the LSST (for a given realization of its cadence)
will be able to carry out its science goals; indeed, we will use this
quantification to refine and optimize the cadence itself. 

The Science Book is six years old.  In those six years, some of the
science themes described in there have evolved or become obselete,
while new science opportunities and ideas have arisen.  Moreover, our
understanding of the capabilities (such as system response and
therefore depth, telescope optics, and so on) have matured
considerably.  The present document endeavors to explore the
principal science themes as described in the Science Book, but is not
slaved to them, and where appropriate, we will point out relevant 
updates to the Science Book.  


% --------------------------------------------------------------------

\section{Synoptic Sky Surveying at Universal Cadence}
\def\secname{intro:baseline}\label{sec:\secname}

Synoptic Surveying with LSST - the basic observing strategy determined
by key projects described in the LSST Science Requirements Document,
and constrained by the LSST's design \citep{IvezicEtal2008}.

% --------------------------------------------------------------------

\section{Beyond the Baseline Observing Strategy}
\def\secname{intro:baseline}\label{sec:\secname}

Optimizing the Observing Strategy: what perturbations are we
permitted to introduce, to maximize the system's science capabilities?
What are our constraints? And our opportunities?

% --------------------------------------------------------------------

\section{Evaluation and Optimization of the LSST Observing Strategy}
\def\secname{intro:evaluation}\label{sec:\secname}

The first step towards a science-based optimization of the LSST
observing strategy is a {\it science-based evaluation of the baseline
LSST observing strategy}. After simulating a sample  observing
schedule consistent with this strategy (see  \autoref{chp:cadexp}), we
then need to quantify its value to each science team. This is what the
LSST DM Sims team's ``Metric Analysis Framework''  was designed to
enable. Once the fiducial strategy has been evaluated in this way,
then any other strategy can be evaluated in the same terms, using the
same code, and we will be able to start optimizing the strategy
through iterations between \OpSim and MAF.

With this program in mind, it makes sense to define {\it one ``Figure
of Merit'' (FoM) per science project}, that captures the value of  the
observing strategy under consideration to that science team. This FoM
will probably be a function of several ``metrics'' that quantify
lower-level features of the observing sequence.  For Figures of Merit
to be directly comparable between disparate science projects,  they
need to be dimensional, and have the same units. One natural
choice could be the {\it information gained} by the science team, in
bits. This is a well-defined statistical quantity, albeit not yet one
in common use. A given observing schedule's value would then depend on
both this information gain, but also {\it how much that information is
worth to the whole community}. It is at this point that the debate
could become heated: probably the best we can do in Cadence Diplomacy
is to quantify all the information gains implied by each proposed
change to the baseline  observing strategy, combine them to see
whether it makes everyone happy, and iterate. In this way we might
hope to minimize the debates about the less quantifiable worth of each
piece of information.

We are some way from being able to define information-based Figures of
Merit for most science cases -- but the metrics that they will depend
on will be easier to derive. Writing this white paper is an
opportunity to think through the Figure of Merit for each science
project that we as a community want to carry out, and how that measure
of success is likely (or even known) to depend on metrics that
summarize the observing sequence presented to us. Thinking about the
problem in terms of science projects, each with a  Figure of Merit,
encourages us to design modular document sections, with one science
project and one Figure of Merit per section.

This will have the happy side-effect of allowing the chapters to be
straightforwardly re-arranged as we go, to make the white paper easier
to read. It will also naturally lead to the definition of a suite of
MAF  super-metrics, can be evaluated on any future \OpSim output
database.  A table in each section showing the values of the metrics
and the FoM, for different schedules, for that science project, will
be very helpful. The metric names in these tables should match the
metric class names in the
\href{https://github.com/LSST-nonproject/sims_maf_contrib/wiki}{\simsMafContrib}
module. In principle these tables could be auto-generated by the MAF
framework, and extended as \OpSim is repeatedly reconfigured and run.

For an example of how all this could look, please see the
\hyperref[sec:lenstimedelays]{lens
time delays section}. The MAF subsections are still under development
there, but keep checking back to see it come together during the
August 2015 workshop week. Templates for the chapters and sections can
be found in \autoref{chp:example}.


% --------------------------------------------------------------------

\section{Outline of This Paper}
\def\secname{intro:outline}\label{sec:\secname}

The rest of this white paper is structured as follows. In
\autoref{chp:cadexp} we describe a number of \OpSim simulated observing
schedules (``cadences'') explored by the LSST Sims team in summer 2015
in preparation for this paper: they include a ``baseline cadence'', and
then some small but interesting perturbations to it. Then, we present
the science cases considered so far, organised into the following
chapters:

\begin{itemize}
    \item \autoref{chp:solarsystem}: \nameref{chp:solarsystem}
    \item \autoref{chp:galaxy}: \nameref{chp:galaxy}
    \item \autoref{chp:astrometry}: \nameref{chp:astrometry}
    \item \autoref{chp:variables}: \nameref{chp:variables}
    \item \autoref{chp:transients}: \nameref{chp:transients}
    \item \autoref{chp:cosmo}: \nameref{chp:cosmo}
    \item \autoref{chp:deepdrilling}: \nameref{chp:deepdrilling}
    \item \autoref{chp:specialsurveys}: \nameref{chp:specialsurveys}
\end{itemize}

Finally, in \autoref{chp:tradeoffs} we bring the results of all the
science metric analyses  together and discuss the tensions between
them, and the trade-offs that we can anticipate having to make.

\navigationbar

% --------------------------------------------------------------------
