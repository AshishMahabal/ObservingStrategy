% --------------------------------------------------------------------

\chapter[Stellar and Binary Variables]{Stellar and Binary Variables}
\def\chpname{variables}\label{chp:\chpname}

Chapter editors:
\credit{AshishMahabal},
\credit{lmwalkowicz}.

% \noindent {\it
% Mike Lund, Ashish Mahabal, Stephen Ridgway,
% Lucianne Walkowicz, Rahul Biswas, Michelle Lochner,
% Jeonghee Rho, Eric Bellm...
% }

% --------------------------------------------------------------------


\section{Introduction}

Variable objects are defined as those that exhibit brightness changes, either periodic or non-periodic, detected in quiescence and which are non-destructive to the object itself. Variable objects span a wide range in timescale-of-interest (sometimes even within a single class of objects), and so different science cases benefit from different sampling strategies. These strategies may be significantly different from one another for different science cases, sometimes even mutually exclusive; competing objectives described in this chapter and the next are at the heart of LSST observing strategy and cadence design.

Below we develop a number of key science cases for LSST studies of variable objects, associating them with related metrics that can be used within the Metrics Analysis Framework (MAF) to understand the impact of a given survey strategy realization on the scientific results for that case. The science cases outlined are by no means exhaustive, but rather are motivated by providing key quantitative examples of LSST's performance given any particular deployment of survey strategy. The authors encourage community contribution of similar cases, where the scientific outcome can be quantified using specific metrics. 



%When evaluating a particular observation or series of observations in
%light of how they perform for a specific science case, it may be
%helpful to think of metrics as lying along a continuum between
%discovery and characterization. Discovery requires a minimum amount of
%information to recognize an event or object as a candidate of
%interest, which necessarily involves some level of bare-bones
%characterization (upon which said recognition is based); rich
%characterization, on the other hand, implies that an event may not
%only be recognized as a candidate of interest, but basic properties of
%the event or object may be determined from the observation (e.g.
%including but not limited to classification of the event). The
%interpretation of a given metric along this continuum has implications
%for the subsequent action and analysis required, particularly as
%regards possible follow-up observations with other facilities.

%Target types are here grouped in subsections by variability
%characteristics, but as will be seen, this does not mean that all
%targets in a group require a common cadence, since the times scales
%may vary dramatically.  Acquiring suitable data for a wide range of
%time scales presents a fundamental problem for LSST, since the
%available $~$800 visits to a field over the survey cannot be deployed
%so as to usefully sample all time scales at all times.  This fact
%leads to the concept of a non-uniform survey, in which parts of the
%sky are visited more frequently part of the time.  The merits of such
%options must be traded against the benefits of a more uniform survey
%strategy.

\subsection{Metrics Employed}
\label{sec:keyword:variablemetrics}


\subsection{Metrics}
\label{sec:keyword:metrics}

\begin{center}
\begin{tabular}{| p{5cm} |p{10cm} |}
\hline Metric & Description\\
\hline
Eclipsing binary discovery & Fraction of discoveries vs fractional duration of eclipse\\
Transiting exoplanets (depth dependent) & Fraction of discoveries vs fractional duration of eclipse\\
Phase gap & Histogram vs period of the median and maximum phase gaps achieved in all fields\\
Period determination (period dependent) & Fraction of targets vs survey duration, for which the period can be determined to 5-sigma confidence\\
Period variability (period dependent) & Fraction of targets vs survey duration, for which a period change of 1\% can be determined with 5-sigma confidence\\
  \hline \end{tabular}
 \end{center}

The period metrics can be based on a standard variable curve (e.g.sinusoid) of fiducial amplitude and brightness, and/or a realistic model population of a particular variable type. These metrics can be informative for science programs.  However, it is not clear that the survey strategy can or should attempt to control these metrics, as the requirements are specific to each target, and all targets benefit from a generally uniform distribution of visits.


Lund et al. (2015; \url{http://arxiv.org/pdf/1508.03175.pdf}) discuss three metrics that have been incorporated into the MAF. Two of these metrics deal explicitly with time variable behavior: a) observational triplets, and b) detection of periodic variability. 

\subsubsection{Periodogram purity function (PeriodicMetric)}
This metric calculates the Fourier power spectral window function of each field (Roberts et al. 1987) as a means of quantifying the completeness of phase coverage for a given periodic variable. The periodogram purity is defined as 1 minus the Fourier power spectral window function; in the perfect case, all power is concentrated in a delta function at the correct frequency, and is zero elsewhere. As power ``leaks'' away from the correct frequency as a consequence of discrete, non-ideal data sampling, the periodogram becomes more structured. For the purposes of MAF metrics, which are designed to quantify performance as a single number, the periodogram purity is quantified as the minimum value away from the correct frequency. 

\subsubsection{Phase Gap Metric (PhaseGapMetric)}
Histogram of the median and maximum phase gaps achieved in all fields

\subsubsection{Period Deviation Metric (PeriodDeviationMetric)}

This metric computes the percent deviation of recovered periods from pure sine variability.

\subsection{Proposed Metrics}

The following is a raw list of metric ideas; these need specificity and further description. 

FWHM of the window function (to quantify sampling)

Maximum hour angle difference 

Fraction of discoveries vs fractional duration of eclipse

Fraction of targets vs survey duration, for which the period can be determined to 5-sigma confidence

Fraction of targets vs survey duration, for which a period change of 1$\%$ can be determined with 5-sigma confidence


\begin{center}
\begin{tabular}{| l | p{8cm} |l | l |}
\hline Periodic Variable Type & Examples of target science & Amplitude & Timescale\\
\hline
Periodic binaries & Eclipses, physical properties of stars, distances, ages, evolution, apsidal precession, mass transfer induced period changes, Applegate effect &  small &  hr-day \\
RR Lyrae & Galactic structure, distance ladder, RR Lyrae properties&  large &  day \\
Cepheids & Distance ladder, cepheid properties&  large &  day \\
Long Period Variables & Distance ladder, LPV properties& large  &  weeks \\
Short period pulsators & Instability strip, white dwarf interior properties, evolution&  small & min  \\
Rotational Modulation & Gyrochronology, stellar activity& small  &  days \\
 \hline \end{tabular}
 \end{center}



% --------------------------------------------------------------------


% ====================================================================
%+
% NAME:
%    section-name.tex
%
% ELEVATOR PITCH:
%    Explain in a few sentences what the relevant discovery or
%    measurement is going to be discussed, and what will be important
%    about it. This is for the browsing reader to get a quick feel
%    for what this section is about.
%
% COMMENTS:
%
%
% BUGS:
%
%
% AUTHORS:
%    Phil Marshall (@drphilmarshall)  - put your name and GitHub username here!
%-
% ====================================================================

\section{Discovery of Periodic Pulsating Variables}
\def\secname{periodicvariables}\label{sec:\secname}

\noindent{\it Lucianne M. Walkowicz, \&c} % (Writing team)

% This individual section will need to describe the particular
% discoveries and measurements that are being targeted in this section's
% science case. It will be helpful to think of a ``science case" as a
% ``science project" that the authors {\it actually plan to do}. Then,
% the sections can follow the tried and tested format of an observing
% proposal: a brief description of the investigation, with references,
% followed by a technical feasibility piece. This latter part will need
% to be quantified using the MAF framework, via a set of metrics that
% need to be computed for any given observing strategy to quantify its
% impact on the described science case. Ideally, these metrics would be
% combined in a well-motivated figure of merit. The section can conclude
% with a discussion of any risks that have been identified, and how
% these could be mitigated.

Regular variables, such as Cepheids and RR Lyraes, are valuable tracers of Galactic structure and cosmic distance. In this case of these and other strictly (or nearly-strictly) periodic variables, data from different cycles of observation can be phase-folded to create a more fully sampled lightcurve as LSST visits will occur effectively at random phases. In a 10-year survey, most periodic stars of almost any period will benefit from excellent phase coverage in all filters (only a very small period range close to the sidereal day will be poorly observed). Therefore, most implementations of the LSST observing strategy will provide good sampling of periodic variables.

However, different implementations of the survey may result in different resulting sample sizes of these periodic variables, and may also affect the environments in which these stars are discovered. In this section, we create a framework for understanding how current implementations of the observing strategy influence (or even bias) the resultant sample size and environments where these important tracers may be identified. 

\subsection{Tracing Galactic Structure with RR Lyrae}

Oluseyi et al. 2012 [INSERT REF] 


% --------------------------------------------------------------------

\subsection{The Cepheid Cosmic Distance Ladder}



% --------------------------------------------------------------------

% --------------------------------------------------------------------

\subsection{OpSim Analysis}
\label{sec:keyword:analysis}

Current simulations show for the main survey a broad uniformity of visits, with thorough randomization of visit phase per period, giving very good phase coverage with minimum phase gaps.


% --------------------------------------------------------------------

\subsection{Discussion}
\label{sec:keyword:discussion}

For periodic variable science, two cadence characteristics should be avoided:
\begin{itemize}
\item an exactly uniform spacing of visits (which is anyway virtually impossible); \
\item a very non-uniform distribution, such as most visits concentrated in a few survey years.
 \end{itemize}

A metric for maximum phase gap will guard against the possibility that a very unusual cadence might compromise the random sampling of periodic variables.

In each case, it would help to jump-start science programs if some fraction of targets had more complete measurements early in the survey.


% ====================================================================

\navigationbar


% --------------------------------------------------------------------


% ====================================================================
%+
% NAME:
%    section-name.tex
%
% ELEVATOR PITCH:
%    Explain in a few sentences what the relevant discovery or
%    measurement is going to be discussed, and what will be important
%    about it. This is for the browsing reader to get a quick feel
%    for what this section is about.
%
% COMMENTS:
%
%
% BUGS:
%
%
% AUTHORS:
%    Phil Marshall (@drphilmarshall)  - put your name and GitHub username here!
%-
% ====================================================================

\section{Probing Planet Populations with LSST}
\def\secname{periodicvariables}\label{sec:\secname}

\noindent{\it Author Name(s)} % (Writing team)

% This individual section will need to describe the particular
% discoveries and measurements that are being targeted in this section's
% science case. It will be helpful to think of a ``science case" as a
% ``science project" that the authors {\it actually plan to do}. Then,
% the sections can follow the tried and tested format of an observing
% proposal: a brief description of the investigation, with references,
% followed by a technical feasibility piece. This latter part will need
% to be quantified using the MAF framework, via a set of metrics that
% need to be computed for any given observing strategy to quantify its
% impact on the described science case. Ideally, these metrics would be
% combined in a well-motivated figure of merit. The section can conclude
% with a discussion of any risks that have been identified, and how
% these could be mitigated.

This section describes the unique discovery space for extrasolar planets with LSST.

\subsection{Planets In Relatively Unexplored Environments}
A large number of exoplanets have been discovered over the past twenty years, with over 1500 exoplanets now confirmed. These discoveries are primarily the result of three detection methods; radial velocity discoveries measure a period and minimum mass of the planet, transiting discoveries measure a radius ratio between the planet and the star and can be combined with RV measurements to constrain the mass and determine a density; microlensing discoveries measure planet mass but represent non-periodic events.

The Kepler mission has an additional almost 4000 planet candidates. While these planet candidates have not been confirmed, the sample is significant enough that planet characteristics can be studied statistically, including radius and period distributions and planet occurrence rates. LSST will add to previous transiting planet searches by observing stellar populations that have generally not been well-studied by previous transiting planet searches, including cluster stars, the galactic bulge, red dwarfs, and the magellanic clouds. Most known exoplanets have been found relatively nearby in the Galaxy, as exoplanet systems with measured distances have a median distance of around 80 pc, and 80\% of these systems are within 320 pc (exoplanets.org). LSST is able to recover transiting exoplanets at much larger distances, including in the galactic bulge and the Large Magellanic Cloud, allowing for measurements of planet occurrence rates in these other stellar environments (Lund http://arxiv.org/pdf/1408.2305v2.pdf, Jacklin http://arxiv.org/pdf/1503.00059v2.pdf). Red dwarfs have often been underrepresented in searches that have focused on solar-mass stars, however red dwarfs are plentiful, and better than 1 in 7 are expected to host earth-sized planets in the habitable zone (Dressing http://arxiv.org/pdf/1501.01623v2.pdf).

While most of the sky that LSST will survey will be at much lower cadences than transiting planet searches employ, a sufficient understanding of the LSST efficiency for detecting planets combined with the large number of targets may still provide significant results. Additionally, the multiband nature of LSST provides an extra benefit, as exoplanet transits are achromatic while many potential astrophysical false positives, such as binary stars, are not.

% --------------------------------------------------------------------

\subsection{OpSim Analysis}
\label{sec:keyword:analysis}



% --------------------------------------------------------------------

\subsection{Discussion}
\label{sec:keyword:discussion}

% ====================================================================

\navigationbar


% --------------------------------------------------------------------


% ====================================================================
%+
% NAME:
%    section-name.tex
%
% ELEVATOR PITCH:
%    Explain in a few sentences what the relevant discovery or
%    measurement is going to be discussed, and what will be important
%    about it. This is for the browsing reader to get a quick feel
%    for what this section is about.
%
% COMMENTS:
%
%
% BUGS:
%
%
% AUTHORS:
%    Phil Marshall (@drphilmarshall)  - put your name and GitHub username here!
%-
% ====================================================================

\section{Age-Mapping the Galaxy Using Gyrochronology}
\def\secname{periodicvariables}\label{sec:\secname}

\noindent{\it Author Name(s)} % (Writing team)

% This individual section will need to describe the particular
% discoveries and measurements that are being targeted in this section's
% science case. It will be helpful to think of a ``science case" as a
% ``science project" that the authors {\it actually plan to do}. Then,
% the sections can follow the tried and tested format of an observing
% proposal: a brief description of the investigation, with references,
% followed by a technical feasibility piece. This latter part will need
% to be quantified using the MAF framework, via a set of metrics that
% need to be computed for any given observing strategy to quantify its
% impact on the described science case. Ideally, these metrics would be
% combined in a well-motivated figure of merit. The section can conclude
% with a discussion of any risks that have been identified, and how
% these could be mitigated.

This section describes recovering stellar rotation periods as a means to mapping ages of stellar populations in the Galaxy.

\subsection{Recovery of Periods from Rotational Modulation}


% --------------------------------------------------------------------

\subsection{OpSim Analysis}
\label{sec:keyword:analysis}



% --------------------------------------------------------------------

\subsection{Discussion}
\label{sec:keyword:discussion}

% ====================================================================

\navigationbar


% --------------------------------------------------------------------


% ====================================================================
%+
% NAME:
%    section-name.tex
%
% ELEVATOR PITCH:
%    Explain in a few sentences what the relevant discovery or
%    measurement is going to be discussed, and what will be important
%    about it. This is for the browsing reader to get a quick feel
%    for what this section is about.
%
% COMMENTS:
%
%
% BUGS:
%
%
% AUTHORS:
%    Phil Marshall (@drphilmarshall)  - put your name and GitHub username here!
%-
% ====================================================================

\section{Discovery and Characterization of Young Stellar Populations}
\def\secname{periodicvariables}\label{sec:\secname}

\noindent{\it Author Name(s)} % (Writing team)

% This individual section will need to describe the particular
% discoveries and measurements that are being targeted in this section's
% science case. It will be helpful to think of a ``science case" as a
% ``science project" that the authors {\it actually plan to do}. Then,
% the sections can follow the tried and tested format of an observing
% proposal: a brief description of the investigation, with references,
% followed by a technical feasibility piece. This latter part will need
% to be quantified using the MAF framework, via a set of metrics that
% need to be computed for any given observing strategy to quantify its
% impact on the described science case. Ideally, these metrics would be
% combined in a well-motivated figure of merit. The section can conclude
% with a discussion of any risks that have been identified, and how
% these could be mitigated.

This section describes the discovery and characterization of young stellar populations using non-periodic variability. 

\subsection{YSOs, FU Oris}


% --------------------------------------------------------------------

\subsection{OpSim Analysis}
\label{sec:keyword:analysis}



% --------------------------------------------------------------------

\subsection{Discussion}
\label{sec:keyword:discussion}

% ====================================================================

\navigationbar


% --------------------------------------------------------------------
