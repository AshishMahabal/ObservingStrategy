\setcounter{chapter}{0}
\chapter*{Preface}
\def\chpname{preface}\label{chp:\chpname}
\addcontentsline{toc}{section}{Preface}
\markboth{}{}

\noindent This is a community white paper outlining various science
cases and the impacts that observing strategy will have on them,
quantified using the Metric Analysis Framework. We will describe
various strategies and tradeoffs that impact the observing cadence
(visit sequence), the current cadence baseline, and future directions
for the optimization of the survey strategy. We aim to publish this
white paper on arXiv, and invite community feedback.

The timescale for producing this white paper, started before and
finished after the Observing Strategy workshop at the  August 2015
LSST Project and Community workshop, is many months.

% --------------------------------------------------------------------

\section*{Messages}
% \addcontentsline{toc}{subsection}{Messages}

The main points we will aim to convey in this white paper are as follows:

\begin{itemize}

    \item We have a pretty good idea of how we would deploy LSST:
    there is a baseline strategy and example cadences, with which it
    can be demonstrated that the data required for the promised
    science can be delivered.

    \item The baseline strategy can and will be optimized -- even small
    improvements can be significant. Most importantly, the strategy is
    not set in stone and it will evolve.

    \item The cadence optimization process will be as open and
    inclusive as technically possible. All stakeholders will
    participate in this process.

\end{itemize}

\raggedright{Project start: July 2015.}

% --------------------------------------------------------------------

\section{Guidelines for Authors}
% \addcontentsline{toc}{section}{Guidelines for Authors}
\def\secname{guidelines}\label{sec:\secname}

\noindent{\it Phil Marshall}
%(\texttt{@drphilmarshall})
\footnote{These notes are pasted from
  \texttt{whitepaper/notes/chapter-template.md}
}

The first section of each science chapter needs to be an {\it introduction}
that outlines, very briefly, the commonality of the key science cases
contained in it:  what is to be measured, in broad-brush terms, and
why this is of interest. Then, suppose we were to design an LSST
survey to enable these measurements: qualitatively, what might it look
like, in terms of the choices we are able to make? This chapter
introduction can eventually (when the results are in!) summarize,
again, in very broad brush terms, the results of a number of
investigative sections, one on each science case.

The individual sections following this introduction will need to
describe the particular discoveries and measurements that are being
targeted in each {\it science case}. It will be helpful to think of a
``science case" as a ``science project" that the section leads {\it
actually plan to do}. Thinking this way means that the sections can
follow the tried and tested format of an observing proposal: a brief
description of the investigation, with references, followed by a
technical feasibility piece.  This latter part will need to be
quantified using the MAF framework, via a set of metrics that need to
be computed for any given observing strategy to quantify its impact on
the described science case. Ideally, these metrics would be combined
in a well-motivated figure of merit. The section can conclude with a
discussion of any risks that have been identified, and how these could
be mitigated.

The following two sections are an example chapter introduction and
science case section for you to work from. The latter is checked into
the repository as \texttt{section-template.tex}.

% --------------------------------------------------------------------

\section{Example Introduction}

General introduction to the chapter's science projects.

Overview of observing strategy needed by those projects, bringing
out common themes or points of tension.

% --------------------------------------------------------------------

% ====================================================================
%+
% SECTION:
%    section-name.tex  % eg lenstimedelays.tex
%
% CHAPTER:
%    chapter.tex  % eg cosmology.tex
%
% ELEVATOR PITCH:
%    Explain in a few sentences what the relevant discovery or
%    measurement is going to be discussed, and what will be important
%    about it. This is for the browsing reader to get a quick feel
%    for what this section is about.
%
% COMMENTS:
%
%
% BUGS:
%
%
% AUTHORS:
%    Phil Marshall (@drphilmarshall)  - put your name and GitHub username here!
%-
% ====================================================================

\section{ Title of Science Project }
\def\secname{keyword}\label{sec:\secname} % For example, replace "keyword" with "lenstimedelays"

\noindent{\it Author Name(s)} % (Writing team)

% This individual section will need to describe the particular
% discoveries and measurements that are being targeted in this section's
% science case. It will be helpful to think of a ``science case" as a
% ``science project" that the authors {\it actually plan to do}. Then,
% the sections can follow the tried and tested format of an observing
% proposal: a brief description of the investigation, with references,
% followed by a technical feasibility piece. This latter part will need
% to be quantified using the MAF framework, via a set of metrics that
% need to be computed for any given observing strategy to quantify its
% impact on the described science case. Ideally, these metrics would be
% combined in a well-motivated figure of merit. The section can conclude
% with a discussion of any risks that have been identified, and how
% these could be mitigated.

A short preamble goes here. What's the context for this science
project? Where does it fit in the big picture?

% --------------------------------------------------------------------

\subsection{Target measurements and discoveries}
\label{sec:keyword:targets}

Describe the discoveries and measurements you want to make.

Now, describe their response to the observing strategy. Qualitatively,
how will the science project be affected by the observing schedule and
conditions? In broad terms, how would we expect the observing strategy
to be optimized for this science?


% --------------------------------------------------------------------

\subsection{Metrics}
\label{sec:keyword:metrics}

Quantifying the response via MAF metrics: definition of the metrics,
and any derived overall figure of merit.


% --------------------------------------------------------------------

\subsection{OpSim Analysis}
\label{sec:keyword:analysis}

OpSim analysis: how good would the default observing strategy be, at
the time of writing for this science project?


% --------------------------------------------------------------------

\subsection{Discussion}
\label{sec:keyword:discussion}

Discussion: what risks have been identified? What suggestions could be
made to improve this science project's figure of merit, and mitigate
the identified risks?


% ====================================================================

\navigationbar


% --------------------------------------------------------------------
